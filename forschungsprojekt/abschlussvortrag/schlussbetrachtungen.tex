\section[Fazit]{Schlussbetrachtungen}
\myContentSectionFrame

%-------------------------------------------------------------------------------

\subsection{Einsatz mehrerer Sensornetze}

%-------------------------------------------------------------------------------

\begin{frame}{\insertsubsection}{}
	\begin{itemize}
	\item 	Nutzer kann vorhandene Geräte einbinden (Kostenminderung)
	\end{itemize}

	\begin{itemize}
	\item 	Jedes Sensornetz kann spezifische Eigenschaften besitzen,
			aber dennoch transparent mit dem Steuerungsserver verbunden sein:
			\begin{enumerate}
			\item 	Energieeffizienz
			\item 	Kommunikationsgeschwindigkeit
			\item 	Authentifizierung (sicherheitskritische Sensorknoten)
			\item 	Einteilung nach logischer oder physikalischer Topologie
			\end{enumerate}
	\end{itemize}
\end{frame}

%-------------------------------------------------------------------------------

%-------------------------------------------------------------------------------

\subsection{Auswertung und Ausblick}

%-------------------------------------------------------------------------------

\begin{frame}{\insertsubsection}{}
	\begin{block}<+->{Constrained Application Protokoll (CoAP)}
		\begin{proconlist}
		\pro 	ist zur Versendung der Sensorwerte geeignet
		\pro 	leichtgewichtig gegenüber HTTP
		\pro 	URIs eignen sich, um logische Strukturen aufzubauen
		\pro 	CoAP-Overhead ist tolerierbar
		\contra CoAP-Proxys zum Cachen von Sensorinformationen nützlich
			(ist aber noch nicht vollständig standardisiert)
		\end{proconlist}
	\end{block}
	\vfill
	\begin{block}<+->{6LoWPAN-Sensornetz}
		\begin{proconlist}
		\pro 	Knoten verbinden sich automatisch mit dem Coordinator
		\contra Authentifizierung (noch) nicht vorhanden
		\contra Verschlüsselung der Nachrichten sollte untersucht werden
		\contra Energieeinsparung kann erhöht werden (MAC/RDC-Protokoll)
		\contra Contiki: Dokumentation in vielen Bereichen dürftig
		\end{proconlist}
	\end{block}
\end{frame}

%-------------------------------------------------------------------------------

\againframe{netzwerkaufbau}
