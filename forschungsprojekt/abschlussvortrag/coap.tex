\subsection[REST]{REST Programmierparadigma im Sensornetz}


\begin{frame}{\insertsubsection}{}
% Warum ist das REST Programmierparadigma sinnvoll im Sensornetz
		\begin{block}<+->{Adressierbarkeit}
			\begin{itemize}
        		\item jeder REST-Dienst eines Servers hat eindeutige Adresse (URI)
        		\item Sensoren und Aktoren haben feste Adresse % unter der sie erreichbar sind
    			\end{itemize}
		\end{block}
		\begin{block}<+->{Zustandslosigkeit}
    			\begin{itemize}
        		\item zustandslose Anfragen
        		\item jeder Request ist in sich geschlossen
        		\item Request enthält alle Informationen für die Verarbeitung
    			\end{itemize}
		\end{block}
		\begin{block}<+->{Operationen}
    			\begin{itemize}
        		\item REST-Server beschreiben ihre Dienste auf Anfrage
        		\item standardisierte Anfragemöglichkeiten\newline
				(GET / PUT / POST / DELETE)
        		\item Sensoren/Aktoren können mit GET abgefragt werden
        		\item Aktoren können mit PUT geschaltet werden
    			\end{itemize}
		\end{block}
\end{frame}



% CoAP : http://datatracker.ietf.org/doc/draft-ietf-core-coap/
\begin{frame}{Constrained Application Protocol (CoAP)}{}
        \begin{proconlist}
            \pro REST Paradigma, aber \enquote{leichtgewichtiger} als HTTP
            \pro einfaches HTTP-Mapping möglich, laut RFC-draft vorgesehen
            \pro optionale Zuverlässigkeitsmechanismen (Confirmable / Non-Confirmable)
            \pro Interoperabilität
            \pro entspricht dem Ansatz \emph{Internet der Dinge}
            \pro geringer Overhead gegenüber
HTTP\footnote{\url{http://www.comnets.uni-bremen.de/itg/itgfg521/aktuelles/fg-workshop-29092011/ITG_HH_thomas_poetsch.pdf}}
            \procon noch in Entwicklung (wird ständig verbessert, jedoch ggf. veraltete Implementierungen)
            \contra größerer Overhead als Eigenentwicklung\newline
			(jedoch mangelnde Interoperabilität)
        \end{proconlist}
\end{frame}

