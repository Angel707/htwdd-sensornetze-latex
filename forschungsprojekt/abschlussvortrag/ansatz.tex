\section[Ansatz]{Ansatz}
\myContentSectionFrame[\thesection - 6]

%----------------------------------------------------------
%----------------------------------------------------------

\subsection[Anforderungen]{Anforderungen an das Hausautomatisierungssystem}

%----------------------------------------------------------

\begin{frame}{\insertsection}{Anforderungen I}
	\begin{block}<+->{Dedizierter Netzknoten zur Regelung und Kontrolle}
		\begin{itemize}
		\item 	für den Hausbewohner leicht konfigurierbar und wartbar
		\item 	Netzknoten befindet sich außerhalb des Sensornetzes
		\end{itemize}
	\end{block}
	\vfill
	\begin{block}<+->{Integration bestehender Sensornetze}
		\begin{itemize}
		\item 	Gewährleistung der Interoperabilität
		\item 	Verringerung der Anschaffungskosten bei vorhandenen Sensornetzen
		\end{itemize}
	\end{block}
\end{frame}

%----------------------------------------------------------

\begin{frame}{\insertsection}{Anforderungen II}
	\begin{block}<+->{Verwendung \enquote{klassischer} Internetprotokolle}
		\begin{itemize}
		\item 	Verwendung auf Vermittlungs-, Transport- und Anwendungsschicht
		\item 	Erhöhung der Transparanz und Annahmebereitschaft der Hausbewohner
		\item 	Einsparung von zusätzlichem Entwicklungsaufwand
				% man betrachte heutzutage die vielen unterschiedlichen Betriebssyteme
				% Mac OS, Windows, Linux Derivate, Android, iOS, BlackBerry, Firefox OS, Chrome OS
		\item 	Analyse klassischer Internetprotokolle\newline
				für den Einsatz in Sensornetzen
				% Batterielebensdauer / Radio Duty Cycle / etc.
		\end{itemize}
	\end{block}
	\vfill
	\begin{block}<+->{Ausfall und Erweiterbarkeit von Netzknoten im Sensornetz}
		\begin{itemize}
		\item 	Wunsch nach Sicherheit (verschlüsselte Übertragung und Authentifizierung)
		\item 	Einfache Installation neuer Sensor-/Aktorknoten
		\item 	Flexibles Routing im Sensornetz
		\end{itemize}
	\end{block}
\end{frame}

%----------------------------------------------------------

\subsection{Verwandte Arbeiten}

%----------------------------------------------------------

	\subsubsection{FHEM}

%----------------------------------------------------------

\begin{frame}{\insertsubsection}{\insertsubsubsection}
	\uncover<1->{\enquote{FHEM ist ein Hausautomations-Server,
		geschrieben von Rudolf Koenig et al. in Perl,
		um FS20-Komponenten und andere Geräte zu steuern.}}
	\uncover<2->{\begin{proconlist}
	\pro 	Dedizierter Netzknoten zur Regelung und Kontrolle
	\pro 	Integration bestehender Sensornetze (FS20, HomeMatic)
	\contra Ausfall und Erweiterbarkeit von Netzknoten:
			\begin{itemize}
			\item 	Regelverarbeitung durch Skriptsprache
			\item 	Komplexe Regeln müssen in Perl implementiert werden
			\end{itemize}
	\contra Verwendung \enquote{klassischer} Internetprotokolle:
			\begin{itemize}
			\item 	es müssen Gateways bereitgestellt werden,
					die beide Netzwerkstacks implementieren
			\item 	keine Transparenz vorhanden
			\item 	Gateways wirken der Interoperabilität entgegen
			\end{itemize}
	\end{proconlist}
	\begin{itemize}
	\item 	Projekt-Seite: \url{http://www.fhemwiki.de/wiki/FHEM}
	\item 	Lizenz: GPLv2
	\end{itemize}}
\end{frame}

%----------------------------------------------------------

	\subsubsection[Hexabus]{Hexabus -- the Kranken of Automation}

%----------------------------------------------------------

\begin{frame}{\insertsubsection}{\insertsubsubsection}
	\uncover<1->{\enquote{An IPv6-based home automation bus.}}
	\uncover<2->{\begin{proconlist}
	\contra 	befindet sich noch im Aufbau
	\contra Kein dedizierter Netzknoten zur Regelung und Kontrolle
			\begin{itemize}
			\item 	Regelverarbeitung verteilt auf verschiedene Knoten
			\item 	erfordert eine Übersicht über alle Knoten beim Anwender
			\end{itemize}
	\contra Keine Integration bestehender Sensornetze
	\pro 	Ausfall und Erweiterbarkeit von Netzknoten durch Routing-Protokoll gelöst
	\pro 	Es werden \enquote{klassische} Internetprotokolle verwendet:
			\begin{itemize}
			\item 	es wird die IPv6-Technologie angewandt: 6LoWPAN
			\item 	Transparenz auf Vermittlungs- und Transportschicht
			\item 	Anwendungsschicht (Hexabus Protokoll): nicht klassisch
			\end{itemize}
	\end{proconlist}
	
	\begin{itemize}
	\item 	Projekt-Seite: \url{https://github.com/mysmartgrid/hexabus/wiki}
	\end{itemize}}
\end{frame}

%----------------------------------------------------------

\subsection{Projekteinteilung}

%----------------------------------------------------------

\begin{frame}{\insertsection}{\insertsubsection}
	\begin{block}{dedizierter Regelungsserver}
		\begin{enumerate}
		\item 	DB zur Verwaltung der Sensoren/Aktoren
		\item 	Integrierung der Anwendungsschicht (CoAP)
		\item 	Regelverarbeitung (inkl. Weboberfläche)
		\end{enumerate}
	\end{block}
	\begin{block}{Gateway zur Integration bestehender Sensornetze}
		\begin{enumerate}
		\item 	FS20
		\item 	HomeMatic
		\end{enumerate}
	\end{block}
	\begin{block}{Eigenes drahtloses Sensornetz (WSN)}
			\begin{enumerate}
			\item 	Border Router
			\item 	Routerknoten
			\item 	Netzknoten mit Sensoren
			\item 	Netzknoten mit Aktoren
			\end{enumerate}
	\end{block}
\end{frame}

%----------------------------------------------------------

%\subsection{Aufgabenstellung}

%----------------------------------------------------------

%\begin{frame}{\insertsubsection}{}
%	\begin{itemize}
%	\item	prüfen, ob mit bestehenden, offenen Technologien der Aufbau
%		einer Heimautomatisierung möglich erscheint
%	\item	als Sensoren und Aktoren sollen Funkmodule genutzt werden
%	\item	Datensicherheit im Konzept von vornherein vorsehen
%	\item	Praxisnähe
%		\begin{itemize}
%		\item	bestehende Technologien kombinieren um Aufwand zu sparen
%		\item	offene Standards verwenden, um anderen die Mitarbeit
%			kostengünstig zu ermöglichen
%		\item	auf Energiesparsamkeit achten, da Knoten mit Batterien
%			laufen
%		\end{itemize}
%	\end{itemize}
%\end{frame}

%-------------------------------------------------------------------------------

% Ausblick
% - Entwicklung eines "neuen" MAC-Protokolls in Contiki, welches dem 6LoWPAN-Standard entspricht
%   und auch energie-effizient arbeitet und einer bestimmten MAC-Protokollgruppe angehört.

