\section{Einführung}
\myContentSectionFrame

%----------------------------------------------------------

\subsection{Gebäude- vs. Hausautomatisierung}

%----------------------------------------------------------

\begin{frame}{Abgrenzung von Haus- und Gebäudeautomatisierung}{Gebäudeautomatisierung}
	\begin{itemize}
	\item 	wichtiger Bestandteil des technischen Facility-Managements
	\item 	Energie- und Personaleinsparungen stehen im Vordergrund
	\item 	dezentrale Anordnung der Steuerungseinheiten
	\item 	Funktionsabläufe gewerkeübergreifend automatisch durchführen
	\item 	Visualisierung der Automation nebensächlich\newline
			(nur in der Managementebene)
	\item 	i.\,d.\,R. herstellerspezifische, kommerzielle Bussysteme
	\item 	Trend zur Offenheit und zum herstellerübergreifenden Informationsaustausch
			(\emph{Interoperabilität}\,)
			% Lockerung der Herstellerabhängigkeit (kein OpenSource-Gedanke!)
			% Bussystem gibt die Möglichkeit, Fabrikate mehrerer Hersteller
			% ohne größere Probleme miteinander kommunizieren zu lassen.
			% Hersteller generieren für ihre Gateways je nach Projekt
			% politische Preise, um der Interoperabilität entgegen zu wirken.
	\end{itemize}
\end{frame}

%----------------------------------------------------------

\begin{frame}{Abgrenzung von Haus- und Gebäudeautomatisierung}{Hausautomatisierung}
	\begin{itemize}
	\item 	Teilbereich der Gebäudeautomatisierung
	\item 	auf spezielle Bedürfnisse der Bewohner von (privaten) Wohnhäusern zugeschnitten
	\item 	erhöhte Wohnkomfort, Sicherheit der Bewohner und Überwachung im Vordergrund
	\item 	Visualisierung und Interaktion der Bewohner besonders wichtig
	\item 	System muss skalierbar sein
	\item 	neue Geräte müssen leicht installiert werden können
	\end{itemize}
\end{frame}

%----------------------------------------------------------

\subsection{Motivation und Ziel}

\begin{frame}{\insertsubsection}{Vorüberlegungen zur Hausautomatisierung}
	\begin{exampleblock}<+->{bestehende Hausautomatisierungslösungen}
	bestehende Heimautomatisierungen besitzen mindestens einen, meist
	mehrere der folgenden, gravierenden Nachteile:
	\begin{itemize}
	\item	teuer
	\item	proprietär
	\item	schlecht dokumentiert
	\item	wenig Sicherheit (\zB in der Datenübertragung)
	\end{itemize}
	\end{exampleblock}

	% neue Netzwerktechnologien (CoAP, 6LoWPAN, \dots) legen die Möglichkeit
	% nahe, mit ihrer Hilfe eine offene Lösung zu konzipieren

	\begin{block}<+->{Wunsch nach \enquote{freier} Hausautomatisierung}
		\begin{itemize}
		\item 	Hersteller-Unabhängigkeit (Interoperabilität)
		\item 	geringere Installationskosten
		\item 	überschaubarer Wartungsaufwand
		\item 	Regelverarbeitung soll möglichst leicht konfigurierbar sein
		\item 	Stichwort \emph{Internet der Dinge}
		\end{itemize}
	\end{block}
\end{frame}

