\section{Einführung}
%\subsection{Themengebiet}
	Die Hausautomatisierung ist ein Teilgebiet der Gebäudeautomatisierung.
	Ein primäres Ziel der Hausautomatisierung ist die finanzielle
	Einsparung, \zB durch intelligente Heizsysteme.
	Zusätzlich soll sie aber auch die Bequemlichkeit der Bewohner fördern.

	Im Bereich der Hausautomatisierung gibt es bereits eine Vielzahl an
	sowohl offenen als auch proprietären Lösungsansätzen.  Jedoch
	erfüllen sie die Bedingungen an eine freie und erweiterbare
	Hausautomatisierungslösung nur bedingt.

	Offene Varianten sind häufig \enquote{Ein-Mann}-Projekte, die
	sehr spezielle Probleme verfolgen oder nur sehr unzureichend
	dokumentiert sind.  Teilweise stützen sie sich aber auch nur darauf,
	vorhandene proprietäre Systeme zu verwalten und miteinander zu
	verbinden.

	Diese proprietären Systeme sind für Fremdanbieter häufig aufgrund hoher
	Lizenzkosten unattraktiv.  Des Weiteren setzen sie Netzwerkprotokolle
	ein, die energieeffizienter als standardisierte Protokolle wie HTTP, TCP
	und IPv6 arbeiten. Die Protokolle stehen aber durch ihre Geschlossenheit
	der Interoperabilität entgegen.


	Diese Probleme begründen die Notwendigkeit des hier vorangetriebenen
	Projektes.

\subsection{Projektziele}
	Im Rahmen des Forschungsprojektes \emph{Sensornetze} an der HTW Dresden,
	geleitet durch Prof. Dr.-Ing. Jörg Vogt, soll überprüft werden, inwiefern
	eine offene Heimautomatisierung durch Anwendung bestehender
	Informationstechnologien in einem Sensornetzwerk realisierbar ist.

	Dabei sind für die Konzipierung im Rahmen unseres Forschungsprojektes
	die folgenden Zielstellungen erarbeitet worden:
	\begin{itemize}
	\item 	Es sollen nur quelloffene und freie Module verwendet werden.
		Dabei ist \enquote{frei} im Sinne der Definition der Free
		Software Foundation zu sehen.
	\item	Das Nutzen existierender, energieeffizienter
		Standardtechnologien \autocite{dunkels04ercim, dunkels08ipso}
		bilden die Grundlage für die
		\emph{Interoperabilität} (Herstellerunabhängigkeit) sowie
		die Langlebigkeit batteriebetriebener Sensorknoten.
	\item 	Die Erweiterbarkeit des Hausautomatisierungssystems
		und damit die Integration anderer -- auch proprietärer --
		Sensornetze ist zu untersuchen.
	\item 	Es ist wünschenswert, dass neue Geräte sich möglichst
		automatisch im System anmelden. Der Ausfall einzelner
		Sensorknoten darf das System nicht gefährden.
		Somit ist eine Skalierbarkeit des Hausautomatisierungssystems
		von Nöten.
	\end{itemize}

\subsection{Projektansatz}
	Die konkret untersuchte Hausautomatisierungslösung besteht aus
	einem dedizierten Netzknoten (Server), der die Regelung des Systems
	übernimmt.
	Auf diesem befindet sich folglich auch die Regelverarbeitung.

	Das Sensornetz soll auf dem Funkstandard 6LoWPAN basieren.
	Als Sensorknoten sollen ATmega128RFA1-Mikrocontroller mit dem
	Betriebssystem Contiki-OS eingesetzt werden.

	Der Informationsaustausch zwischen den einzelnen Sensorknoten und dem
	Server soll über das Constrained Application Protocol (CoAP) realisiert
	werden.
	CoAP ist ein im Vergleich zu HTTP energieeffizientes Protokoll, welches
	dennoch URIs unterstützt.

	Bestehende Sensornetze, wie FS20 oder HomeMatic,
	werden mit Hilfe von Gateways integriert,
	welche ein Mapping zwischen dem entsprechenden Sensornetzprotokoll
	und CoAP durchführen.

\subsection{Projekteinteilung}
	Zur Konzipierung der offenen Hausautomatisierung wurde
	das Projekt in folgende Teilprojekte gegliedert:
	\subsubsection{Steuerung des Systems}
		Es gilt zu untersuchen, ob ein zentraler oder dezentraler
		Hausautomatisierungsansatz in Hinblick auf die Bedürfnisse
		der Nutzer des Systems sinnvoll ist. Danach ist
		die Steuerung des Systems zu entwickeln
		und ein Anwendungsprotokoll zu finden, das sowohl im Sensornetz
		als auch auf normalen Computern eingesetzt werden kann.
	\subsubsection{Regelverarbeitung}
		Sofern die Steuerung des Systems definiert ist, kann
		mit der Regelverarbeitung begonnen werden.
		Hierbei sind die Fähigkeiten verschiedener Nutzer
		besonders zu berücksichtigen und eine einfach konfigurierbare
		Regelverarbeitung zu finden.
	\subsubsection{Sensornetz-Kommunikation}
		Neben der Steuerung muss die Kommunikation im Sensornetz
		analysiert werden und es muss ein Netzwerkstack gefunden
		werden, der sowohl die Anforderungen an das
		Hausautomatisierungssystems erfüllt
		wie auch diejenigen Anforderungen an
		\emph{Tiny-Networked-Sensors} \autocite{dunkels04contiki}.
	\subsubsection{Sensoransteuerung}
		\label{sec:teilprojekt}
		Um im Netzwerk Sensordaten auswerten zu können, muss untersucht
		werden, wie Sensoren und Aktoren auf dem im Ansatz beschriebenen
		Sensorknoten angesteuert und an die CoAP-Schnittstelle
		weitergeleitet werden können.
	\subsubsection{Integration bestehender Sensornetze}
		Wie in den Zielstellungen definiert, sollen grundsätzlich
		bestehende Sensornetze verwendet werden können.
		Beispielhaft wurden hierbei FS20- und HomeMatic-Geräte in das
		Hausautomatisierungssystem eingebunden.

\subsection{Dokumentabgrenzung}
	Dieses Dokument soll sich lediglich mit dem Teilprojekt der
	Sensoransteuerung beschäftigen.
	
	Dabei soll dargestellt werden, welche Möglichkeiten zur
	Ansteuerung der Peripherie existieren.	Des Weiteren sollen die
	Möglichkeiten unter Contiki untersucht werden und die Auswirkungen
	auf das Kommunikationsverhalten des Sensorknotens dargestellt werden.

	Der Einsatz von CoAP als Kommunikationsprotokoll wird dabei nicht
	detailliert untersucht. Es wird jedoch als Rahmenbedingung für den
	Einsatzzweck herangezogen.

	Durch den Einsatz der batteriebetriebenen ATmega128RFA1-Mikrocontroller
	als Sensorknoten müssen insbesondere die knappen Ressourcen Speicher und
	Energie beachtet werden.
