%%%%%%%%%%%%%%%%%%%%%%%%%%%%%%%%%%%%%%%%%%%%%%%%%%%%%%%%%%%%
%% Lokalisierung %%%%%%%%%%%%%%%%%%%%%%%%%%%%%%%%%%%%%%%%%%%
%%%%%%%%%%%%%%%%%%%%%%%%%%%%%%%%%%%%%%%%%%%%%%%%%%%%%%%%%%%%
\usepackage[utf8]{inputenc}	% Umlaute direkt eingeben
\usepackage[T1]{fontenc}	% Wörter mit Umlaute umbre-
				% chen
\usepackage[ngerman]{babel}		% deutsche Bezeichner
\usepackage[babel,german=quotes]{csquotes}	% \enquote{}
%\usepackage{libertine}
\usepackage[scaled=0.83]{beramono}
\usepackage{microtype}

%%%%%%%%%%%%%%%%%%%%%%%%%%%%%%%%%%%%%%%%%%%%%%%%%%%%%%%%%%%%
%% IEEEtran tools %%%%%%%%%%%%%%%%%%%%%%%%%%%%%%%%%%%%%%%%%%
%%%%%%%%%%%%%%%%%%%%%%%%%%%%%%%%%%%%%%%%%%%%%%%%%%%%%%%%%%%%
%\usepackage{IEEEtrantools}

%%%%%%%%%%%%%%%%%%%%%%%%%%%%%%%%%%%%%%%%%%%%%%%%%%%%%%%%%%%%
%% Tabellen %%%%%%%%%%%%%%%%%%%%%%%%%%%%%%%%%%%%%%%%%%%%%%%%
%%%%%%%%%%%%%%%%%%%%%%%%%%%%%%%%%%%%%%%%%%%%%%%%%%%%%%%%%%%%
\usepackage{tabularx}
\usepackage{booktabs}	% \toprule\midrule\bottomrule
			% \addlinespace
\usepackage{threeparttable} % Tablenotes
\usepackage{rccol}

%%%%%%%%%%%%%%%%%%%%%%%%%%%%%%%%%%%%%%%%%%%%%%%%%%%%%%%%%%%%
%% Bilder %%%%%%%%%%%%%%%%%%%%%%%%%%%%%%%%%%%%%%%%%%%%%%%%%%
%%%%%%%%%%%%%%%%%%%%%%%%%%%%%%%%%%%%%%%%%%%%%%%%%%%%%%%%%%%%
\usepackage{graphicx}	% \includegraphics{bild.pdf}
\usepackage{rotating}
\usepackage{tikz}
\usetikzlibrary{arrows}
\usetikzlibrary{shapes}
\usetikzlibrary{fit}
\usetikzlibrary{backgrounds}
\usetikzlibrary{datavisualization}
\usepackage{circuitikz}

% Paket zur Anpassung von Titeln von Gleitobjekten
%\usepackage[figureposition=bottom,tableposition=above]{caption}
% Tabelle 1: ABC
%            XYZ
%\captionsetup{format=hang}


% Paket, um Grafiken / Tabellen zu gruppieren (subfigure is obsolete)
% ---> Workaround
% The problem is that there does not seem to be a way to prevent subcaption
% from taking control of the main caption formatting away from IEEEtran like
% the caption=false option does under subfig.sty. IEEEtran has to format
% captions differently depending on its mode.
%\makeatletter
%\let\MYcaption\@makecaption
%\makeatother
%\usepackage[font=footnotesize]{subcaption}
%\makeatletter
%\let\@makecaption\MYcaption
%\makeatother
% see fig/subfigure-example.tex

%%%%%%%%%%%%%%%%%%%%%%%%%%%%%%%%%%%%%%%%%%%%%%%%%%%%%%%%%%%%
%% Farben %%%%%%%%%%%%%%%%%%%%%%%%%%%%%%%%%%%%%%%%%%%%%%%%%%
%%%%%%%%%%%%%%%%%%%%%%%%%%%%%%%%%%%%%%%%%%%%%%%%%%%%%%%%%%%%
\usepackage{xcolor}
% hard colors
\definecolor{hllblue}{rgb}{.2,.2,.7}
% symbolic colors
\definecolor{todocolor}{named}{red}
\definecolor{linkcolor}{named}{hllblue}
\definecolor{lstbg}{gray}{.9}

%%%%%%%%%%%%%%%%%%%%%%%%%%%%%%%%%%%%%%%%%%%%%%%%%%%%%%%%%%%%
%% Mathematische Symbole %%%%%%%%%%%%%%%%%%%%%%%%%%%%%%%%%%%
%%%%%%%%%%%%%%%%%%%%%%%%%%%%%%%%%%%%%%%%%%%%%%%%%%%%%%%%%%%%
\usepackage{amssymb}
\usepackage{amsmath}
\usepackage{amsfonts}

%%%%%%%%%%%%%%%%%%%%%%%%%%%%%%%%%%%%%%%%%%%%%%%%%%%%%%%%%%%%
%% Sonstige Symbole %%%%%%%%%%%%%%%%%%%%%%%%%%%%%%%%%%%%%%%%
%%%%%%%%%%%%%%%%%%%%%%%%%%%%%%%%%%%%%%%%%%%%%%%%%%%%%%%%%%%%
\usepackage{eurosym}
\usepackage{xspace}
\usepackage{textcomp}
\usepackage[italian,	% \unita, conflict with babel
	squaren,	% \squaren, conflict with amssymb
	binary		% \byte usw.
	]{SIunits}

%%%%%%%%%%%%%%%%%%%%%%%%%%%%%%%%%%%%%%%%%%%%%%%%%%%%%%%%%%%%
%% Quellcodes %%%%%%%%%%%%%%%%%%%%%%%%%%%%%%%%%%%%%%%%%%%%%%
%%%%%%%%%%%%%%%%%%%%%%%%%%%%%%%%%%%%%%%%%%%%%%%%%%%%%%%%%%%%
\usepackage{listings}
%\usepackage{scrhack}	% Warnungen im Zusammenhang mit
			% listings verhindern, zusätzlich
			% für Listings das Babelpaket
			% aktivieren, usw.
\lstdefinestyle{nummeriert}{numbers=left}
\lstdefinestyle{monospace}{basicstyle=\ttfamily\scriptsize}
\lstdefinestyle{block}{style=monospace,
	emphstyle=\bfseries,
	breaklines=true,
	breakatwhitespace=false,
	prebreak=\raisebox{0ex}[0ex][0ex]{\ensuremath{\hookleftarrow}}
	}
\lstdefinestyle{float}{style=block,float}
%\lstdefinestyle{block}{style=nummeriert,%
	%style=monospace,%
	%backgroundcolor=lstbg}
\lstset{basicstyle=\ttfamily}

%%%%%%%%%%%%%%%%%%%%%%%%%%%%%%%%%%%%%%%%%%%%%%%%%%%%%%%%%%%%
%% pdf-links %%%%%%%%%%%%%%%%%%%%%%%%%%%%%%%%%%%%%%%%%%%%%%%
%%%%%%%%%%%%%%%%%%%%%%%%%%%%%%%%%%%%%%%%%%%%%%%%%%%%%%%%%%%%
\usepackage{varioref}	% \vpageref{}
\usepackage[%
	pdftex,	% in Links Umbrüche erlauben
	bookmarks,
	bookmarksopen=false
	]{hyperref}
\hypersetup{
	pdfauthor={Angelos Drossos, Hermann Lorenz},
	pdftitle={Ansteuerung der Sensoren und Aktoren in einem 6LoWPAN-Sensornetz}
	}	% \autoref{}
\hypersetup{%
	%ocgcolorlinks,	% beim Drucken die Linkfarben ignorieren
			% kann unter Umständen Probleme bereiten
			% Links können nicht umgebrochen werden
	colorlinks,%
	linkcolor=linkcolor,%
	urlcolor=linkcolor,%
	linktoc=all	% in Verzeichnissen Zahlen und Text verlinken
	}


%%%%%%%%%%%%%%%%%%%%%%%%%%%%%%%%%%%%%%%%%%%%%%%%%%%%%%%%%%%%
%% Fortsetzung Symbole %%%%%%%%%%%%%%%%%%%%%%%%%%%%%%%%%%%%%
%%%%%%%%%%%%%%%%%%%%%%%%%%%%%%%%%%%%%%%%%%%%%%%%%%%%%%%%%%%%
%\usepackage{ellipsis}	% solve \dots problems


%%%%%%%%%%%%%%%%%%%%%%%%%%%%%%%%%%%%%%%%%%%%%%%%%%%%%%%%%%%%
%% Literaturverzeichnis %%%%%%%%%%%%%%%%%%%%%%%%%%%%%%%%%%%%
%%%%%%%%%%%%%%%%%%%%%%%%%%%%%%%%%%%%%%%%%%%%%%%%%%%%%%%%%%%%
% recommend:
%\usepackage[ngerman]{babel}
%\usepackage[babel,german=quotes]{csquotes}

% Bibliographien erstellen mit biblatex (Teil 1)
% see: http://biblatex.dominik-wassenhoven.de/download/DTK-2_2008-biblatex-Teil1.pdf

\usepackage[%
	backend=biber,    % (bibtex, biber)
	%bibencoding=utf8, % wenn .bib in utf8, sonst ascii
	%safeinputenc,     % Das inputenc Package hat nur einen begrenzten UTF8-Satz
	%bibwarn=true,     % Warnung bei fehlerhafter bibDatei
	%sortlocale=de,    % Deutsche Sortierung aktivieren
	%style=alphabetic, % Zitierstil [alphabetic, numeric-comp, etc.]
	style=ieee,       % Zitierstil IEEE, ab biblatex-1.2c
	%style=ieee-alphabetic,       % Zitierstil IEEE, ab biblatex-1.2c
	%isbn=false,                % ISBN nicht anzeigen, gleiches geht
	%                           %    mit nahezu allen anderen Feldern
	%pagetracker=true,          % ebd. bei wiederholten Angaben
	%                           %    (false=ausgeschaltet, page=Seite,
	%                           %    spread=Doppelseite, true=automatisch)
	%maxbibnames=50,            % maximale Namen, die im Literaturverzeichnis
	%                           %    angezeigt werden (ich wollte alle)
	%maxcitenames=3,            % maximale Namen, die im Text angezeigt werden,
	%                           %    ab 4 wird u.a. nach den ersten Autor angezeigt
	%autocite=inline,           % regelt Aussehen für \autocite (inline=\parancite)
	%block=space,               % kleiner horizontaler Platz zwischen den Feldern
	%backref=true,              % Seiten anzeigen, auf denen die Referenz vorkommt
	%backrefstyle=three+,       % fasst Seiten zusammen, z.B. S. 2f, 6ff, 7-10
	%date=short,                % Datumsformat
]{biblatex}

% Abstände
%\setlength{\bibitemsep}{1em}   % Abstand zwischen den Literaturangaben
%\setlength{\bibhang}{2em}      % Einzug nach jeweils erster Zeile

% Bibtex-Datei wird schon in der Preambel eingebunden
\addbibresource{literature.bib} % Biber
%\bibliography{liteature}        % BibTex

% bibtex:
% \usepackage{cite}

% BibLatex/Biber Commands aufheben (weil Biber nicht installiert ist)
%\usepackage{xargs}
%\renewcommandx{\cite}[3][1=, 2=]{\textcolor{red}{#1 [CITE(#3)] #2}}
%\renewcommandx{\Cite}[3][1=, 2=]{\textcolor{red}{#1 [CITE(#3)] #2}}
%\renewcommandx{\autocite}[3][1=, 2=]{\textcolor{red}{#1 [AUTOCITE(#3)] #2}}
%\renewcommandx{\Autocite}[3][1=, 2=]{\textcolor{red}{#1 [AUTOCITE(#3)] #2}}
%\renewcommandx{\parencite}[3][1=, 2=]{\textcolor{red}{(#1 [PARENCITE(#3)] #2)}}
%\renewcommandx{\Parencite}[3][1=, 2=]{\textcolor{red}{(#1 [PARENCITE(#3)] #2)}}
%\renewcommandx{\citetitle}[3][1=, 2=]{\textcolor{red}{#1 CITETITLE(#3) #2}}

%%%%%%%%%%%%%%%%%%%%%%%%%%%%%%%%%%%%%%%%%%%%%%%%%%%%%%%%%%%%
%% Abkürzungsverzeichnis %%%%%%%%%%%%%%%%%%%%%%%%%%%%%%%%%%%
%%%%%%%%%%%%%%%%%%%%%%%%%%%%%%%%%%%%%%%%%%%%%%%%%%%%%%%%%%%%

%\usepackage[% Optionen
	%footnote, % die Langform als Fußnote ausgeben
	%nohyperlinks, % Wenn hyperref geladen ist, wird die Verlinkung unterbunden
%	printonlyused, % nur Abkürzungen auflisten, die tatsächlich verwendet werden
	%	withpage, % im printonlyused-Modus: Seitenzahl der ersten Verwendung angeben
	%smaller, % Text soll kleiner erscheinen (Package relsize vorausgesetzt)
	%dua, % es wird immer die Langform ausgegeben
	%nolist, % es wird keine Liste mit allen Abkürzungen ausgegeben
%	]{acronym}

%%%%%%%%%%%%%%%%%%%%%%%%%%%%%%%%%%%%%%%%%%%%%%%%%%%%%%%%%%%%
%% eigene Macros %%%%%%%%%%%%%%%%%%%%%%%%%%%%%%%%%%%%%%%%%%%
%%%%%%%%%%%%%%%%%%%%%%%%%%%%%%%%%%%%%%%%%%%%%%%%%%%%%%%%%%%%
\newcommand{\vautoref}[1]{\autoref{#1}\vpageref{#1}}
\newcommand{\todo}[1]{%
	\marginpar{\footnotesize\textcolor{todocolor}{TODO}}%
	\textcolor{todocolor}{TODO: #1}%
	}
\newcommand{\todox}[1]{%
	\marginpar{\footnotesize\textcolor{todocolor}{#1}}%
	\textcolor{todocolor}{#1}%
	}
\newcommand{\zB}{z.\,B.\xspace}
\newcommand{\dhx}{d.\,h.\xspace}
\newcommand{\idR}{i.\,d.\,R.\xspace}
\newcommand{\iA}{i.\,A.\xspace}
\newcommand{\uU}{u.\,U.\xspace}
\newcommand{\ItC}{I\textsuperscript{2}C\xspace}

% use: Hallo (\engl{hello})
\newcommand{\engl}[1]{engl.\,\emph{#1}\,\xspace}

% \contikifile{path/to/file}{version}
% use: \contikifile{core/dev/radio.h}{2.6}
\newcommand{\contikifile}[2]{\emph{#1}}

\newcommand*{\changefont}[3]{\fontfamily{#1}\fontseries{#2}\fontshape{#3}\selectfont}
\newcommand{\func}[1]{{\changefont{cmtt}{m}{sc}#1}}
\newcommand{\event}[1]{{\changefont{cmtt}{m}{sc}#1}}

\newcommand{\theadhll}[1]{\emph{\scriptsize #1}}

\newcommand{\tabref}[1]{Tabelle~\ref{tab:#1}}
\newcommand{\figref}[1]{Abbildung~\ref{fig:#1}}
\newcommand{\secref}[1]{Kapitel~\ref{sec:#1}}


\newcommand{\siehe}[1]{siehe \autoref{#1}}
\newcommand{\parensiehe}[1]{(\siehe{#1})}

\newcommand{\ieeeframe}{IEEE~802.15.4}
\newcommand{\thead}[1]{\itshape\scriptsize#1}

%%%%%%%%%%%%%%%%%%%%%%%%%%%%%%%%%%%%%%%%%%%%%%%%%%%%%%%%%%%%
%% PDF Bookmarks %%%%%%%%%%%%%%%%%%%%%%%%%%%%%%%%%%%%%%%%%%%
%%%%%%%%%%%%%%%%%%%%%%%%%%%%%%%%%%%%%%%%%%%%%%%%%%%%%%%%%%%%
\makeatletter
\usepackage{etoolbox}
\pretocmd{\tableofcontents}{%
	\if@openright\cleardoublepage\else\clearpage\fi
	\pdfbookmark[0]{\contentsname}{toc}%
}{}{}%
\makeatother

% Trennmuster
\hyphenation{IEEE Be-triebs-span-nung}
