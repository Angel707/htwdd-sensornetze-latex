\section{Einführung in Contiki}

% *1. Einführung in Contiki* (ca. 10min)
%
% - Paper: Operating Systems for Wireless Sensor Networks: A Survey
%          Muhammad Omer Farooq and Thomas Kunz (2011)
%          www.mdpi.com/1424-8220/11/6/5900/pdf

% - insb. für die Leute, die von Contiki noch nichts gehört haben.
% - Was ist Contiki
% - Wofür ist es primär gedacht (Internet)?
% - Idee der Umsetzung (Implementierungsansatz: Trennung von
%   core/cpu/platform, etc..)
% - Welche Lizenz? (3-clause BSD-style license, aber ohne die Pflicht, den
%   Code OpenSource zu stellen.)

%-------------------------------------------------------------------------------
\begin{frame}{Was ist Contik-OS?}
	\begin{itemize}
	\item	Open-Source-Betriebssystem (3-Klausel-BSD-Lizenz)
	\item	für
		\begin{itemize}
		\item	eingebettete, batteriebetriebene Systeme
		\item	Netzwerkgeräte
		%\item	bereits implementierte Netzwerkstacks
		\end{itemize}
	\item	bereits auf verschiedenste Hardwareplattformen portiert
	\end{itemize}

	\uncover<2->{
	\begin{center}
	\alert{\enquote{Open Source Operation System for the Internet of Things}}
	\end{center}
	}
\end{frame}
%....... note .................................................................
\note{
	Contiki-OS ist ein OpenSource Betriebssystem, welches ermöglicht,
	dass eingebettete Systeme miteinander und mit Rechnersystemen
	über Internettechnologien kommunizieren können.
	\bigskip

	Alle Modifikationen am Betriebssystem werden mit einer 3-Klausel-BSD-Lizenz
	gemacht, wodurch eine Firma/Person die Möglichkeit hat, seinen Änderungen
	geheim zu halten.
	\bigskip

	Contiki ist vor allem für eingebettete, also leistungsarme, Systeme
	entworfen worden. Zumeist werden diese Systeme mit Batterien betrieben,
	wodurch Schlafzeiten länger als Wachzeiten dauern sollten.
	\bigskip

	Da Contiki mit der Außenwelt kommunizieren möchte, muss man sich überlegen,
	wie man diese sich weiter entwickelnden Technologien sinnvoll
	integrieren kann. Genau dieses stellen wir EUCH \emph{übernächstes} Kapitel
	vor.
}
\note{
	Da Contiki bereits implementierte Netzwerk-Stacks bietet,
	können sofort Anwendungen geschrieben werden, die mit der Außenwelt
	kommunizieren.
	\bigskip

	Ein gutes Beispiel für die funktionierende Kommunikation
	ist ein kleiner Webserver, welcher letztes Semester im Rahmen des
	Forschungsseminars vorgestellt wurde.
	\bigskip

	Wer allerdings die entsprechende Hardware nicht zur Verfügung hat,
	kann zur Netzwerk-Simulation mittels Cooja zurückgreifen.
}
%-------------------------------------------------------------------------------
\begin{frame}{Welche Funktionalität bietet Contiki?}
	\begin{itemize}
	\item	Netzwerkunterstützung: UDP, TCP, HTTP, 6LoWPAN, RimeStack, RPL, CoAP
	\item	Protothreads
	\item	Speicherverwaltung:
		\begin{itemize}
		\item	memory block allocation (memb)
		\item	managed memory allocator (mmem)
		\end{itemize}
	\item	Dateisystem: Coffee flash file system (CFS)
	\item	Programme zur Laufzeit nachladen \conclusion \enquote{Dynamic Module Loading}
	\item	Tools:
		\begin{itemize}
		\item	Netzwerk-Simulation durch den Cooja Simulator
		\item	Überwachung des Energieverbrauchs (EnergEst-Modul)
		\end{itemize}
	\end{itemize}
\end{frame}
%....... note .................................................................
\note{
	Die Funktionsvielfalt von Contiki hat in den letzten Jahren enorm
	zugenommen. Da wir im Rahmen dieses Vortrags nicht alle Einzelheiten
	ansprechen können, haben wir uns auf eine kleine Menge beschränkt.
	\bigskip

	Dennoch möchten wir zuerst einen Überblick zu den Funktionen
	geben:

	\noteparagraph{Protothreads}
	Ein wirklich grundlegendes Konzept in Contiki sind die Protothreads,
	die wir im \emph{nächsten} Kapitel näher beleuten werden.
	Diese ermöglichen die Verwendung von Threads, welche kooperativ arbeiten.
	Da sie kooperativ arbeiten, ist kein seperater Stack für jeden Thread
	von nöten.

	\noteparagraph{Coffee flash file system (CFS)}
	Contiki bietet die Möglichkeit, Dateien zu verwalten --
	und zwar unabhängig von der verwendeten
	Hardware (SRAM, ROM, EEPROM, etc.).
}
\note{
	\noteparagraph{Memory Management}
	Jeder Programmierer kennt das Problem, wie man dynamisch Speicher
	allokiert. Besonders in eingebetteten Systemen sollte man möglichst wenig
	\emph{malloc} verwenden, da es zur Speicherfragmentierung führt und
	es bei falscher Verwendung schnell zu Speicherlecks sowie
	undefiniertes Verhalten kommt.
	\smallskip

	Hierzu hat Contiki gleich zwei Technologien parat:
	Zum einen \emph{Memory block allocation (memb)}, welches einen statischen
	Speicherbereich in gleichgroße Blöcke einteilt,
	und zum anderen den \emph{Managed Memory Allocator (mmem)}, der
	den freien Speicherbereich zusammenführt, wodurch Programme nie sicher sein
	können, dass der allokierte Speicher am selben Platz bleibt.
}
\note{
	\noteparagraph{Full IP Networking}
	Wie bereits angesprochen, gibt es eine Reihe von bereits implementierten
	Internettechnologien. Einige sind besonders für Systeme mit sehr wenig
	Speicher entwickelt worden -- wie wir bereits in der Vorlesung erfahren
	haben: 6LoWPAN.

	\noteparagraph{Energest Modul}
	Bei Betriebssystemen einer bestimmten Größe verliert man schnell
	die Übersicht, wie lange oder wie oft ein bestimmtes Ereignis 
	ausgeführt wird. Um den Arbeitsablauf des eingebetteten Systems
	optimieren zu können, ist das Energest Modul entwickelt worden.
	Es zählt wie ein Counter mit, wie lange etwas dauert -- wie beim
	Ein- und Auschecken in Firmengebäuden. Dadurch kann man sehen, an welchen
	Stellen man optimieren sollte.

	\noteparagraph{Dynamic Module Loading}
	Dynamisches Starten von Modulen bzw. Programmcode ist die wichtigste
	Funktion in modernen Betriebssystemen und ist damit auch auf eingebetteten
	Systemen durchaus von Vorteil.
	Dieses werden wir im \emph{letzten} Kapitel näher beleuten.
}
%-------------------------------------------------------------------------------
\begin{frame}{Wie ist Contiki aufgebaut?}
	\begin{tikzpicture}
	\tikzstyle{subele}=[draw,rounded corners,top color=white,bottom color=hllblue!50,very thick,draw=hllblue!50!black!50];
	\tikzstyle{ele}=[draw,rounded corners,top color=white,bottom color=hllorange!50,very thick,draw=hllorange!50!black!50];

	\node [ele] at (0,5.6) [minimum width=10cm,minimum height=.8cm] {};
	\node at (-3.7,5.6) {Applikationen};

	\node [ele] at (0,4) [minimum width=10cm,minimum height=1.5cm] {};
	\node at (-4.2,4.5) {Services};
	\node [subele] at (-3.5,4) {Webserver};
	\node [subele] at (-1.6,4) {FTP};
	\node [subele] at (0,4) {Telnet};
	\node [subele] at (1.7,4) {\phantom{A}\dots\phantom{A}};
	\node [subele] at (3.7,4) {eigene};

	\node [ele] at (0,2) [minimum width=10cm,minimum height=1.5cm] {};
	\node at (-4.2,2.5) {Kern};
	\node [subele] at (-3.7,2) {Process};
	\node [subele] at (-2.0,2) {Timer};
	\node [subele] at (-.2,2) {Netzwerk};
	\node [subele] at (2,2) {ELF-Loader};
	\node [subele] at (4,2) {\phantom{A}\dots\phantom{A}};

	\node [ele] at (0,0) [minimum width=10cm,minimum height=1.5cm] {};
	\node at (-4.2,.5) {Treiber};
	\node [subele] at (-2.5,0) [minimum width=4cm] {Plattform};
	\node [subele] at (2.5,0) [minimum width=4cm] {CPU};

	\node [ele] at (6,2.6) [minimum width=6.8cm,rotate=90] {Tools};
\end{tikzpicture}

	%\todo{Contiki-Schichtenmodell von Marcus Dreier übernehmen/abwandeln??}
	%\prove{DA Portierung des Betriebssystems Contiki auf die Blackfin Architektur.}
	%\url{http://www2.htw-dresden.de/~robge/arbeiten/2010/da-marcus-dreier.pdf}
\end{frame}
%-------------------------------------------------------------------------------
%\begin{frame}{Die Contiki-OS--Architektur}{Verzeichnis}
%	\begin{itemize}
%	\item	apps (telnet,
%			shell,
%			\dots)
%	\item	core
%	\item	cpu (arm,
%			avr,
%			z80,
%			\dots)
%	\item	doc
%	\item	examples
%	\item	platform (avr-raven,
%			avr-atmega128rfa1,
%			\dots)
%	\item	tools
%	\end{itemize}
%\end{frame}
%-------------------------------------------------------------------------------
