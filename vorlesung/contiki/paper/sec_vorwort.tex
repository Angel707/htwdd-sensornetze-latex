\chapter*{\pdfbookmark[0]{Vorwort}{vorwort}Vorwort}
	Das von Professor Vogt betreute Forschungsseminar \enquote{Sensornetze}
	hat als Zielstellung die Konzipierung einer \emph{freien}
	Hausautomatisierung.  In diesem Bereich existieren schon verschiedene
	proprietäre (\zB ZigBee und HomeMatic) und auch freie Lösungsansätze
	(\zB Hexabus), die aber die Anforderungen an eine freie,
	erweiterbare Hausautomatisierung nicht erfüllen.

	\medskip

	Proprietäre Lösungen haben den Nachteil, dass sie aufgrund
	der geschlossenen Technik oder aufgrund unnötiger Lizenzkosten
	für Drittanbieter unattraktiv sind.
	
	Freie Systeme sind meist Ein-Mann-Projekte, die zu spezielle
	Lösungsansätze verfolgen oder nur unzureichend dokumentiert sind.

	\medskip

	Ziele für die Entwicklungen im Forschungsseminar sind:
	\begin{itemize}
	\item	quelloffene Module
	\item	existierende Standardtechnologien nutzen, wie \acf{6LoWPAN}
	\item	ausführliche Dokumentationen
	\item	die Möglichkeit andere, evtl. auch proprietäre, Systeme
		mit einzubinden
	\end{itemize}

	\medskip

	Im Rahmen dieser Arbeit für das Mastermodul Sensornetze soll -- im
	Bezug auf das Forschungsseminar --  das freie Betriebssystem Contiki-OS
	analysiert und vorgestellt werden.

	Da für das Forschungsseminar der Einsatz von \acs{6LoWPAN} als
	Funktechnologie und dem
	ATmega128RFA1 als Hardwareplattform vorgesehen ist, werden in diesem
	Dokument vorwiegend die im Forschungsseminar eingesetzten Technologien
	betrachtet. Andere Funktechnologien werden nur beiläufig erwähnt.
	Dennoch soll insbesondere aus dem \autoref{sec:NetStack} hervorgehen,
	wie neue Funktechnologien integriert werden können.

