\label{sec:NetStack}
	Dieses Kapitel behandelt den \ac{NetStack} und seine
	Implementationsweise.

	Dieser nimmt in Contiki-OS eine besondere Stellung ein, weswegen die
	Anfordrungen an den \ac{NetStack} zuerst behandelt werden.  Da die
	Kommunikation der Sensorknoten untereinander bestimmte Eigenschaften
	erfordert, werden im folgenden Abschnitt \emph{Grundlagen der
	Kommunikation in Sensornetzen} vermittelt.  Danach wird das Konzept
	bzw. der Aufbau des \ac{NetStack}s gezeigt sowie deren Funktionalitäten
	erläutert.  Zuletzt wird der Ablauf beim Versenden und Empfangen einer
	Nachricht beispielhaft demonstriert.


	Bei der Analyse des Contiki NetStacks ist vor allem der Quelltext von
	Contiki \autocite{Contiki:GitHub:2.6rc0} analysiert worden, da der
	Ablauf aus diesem am besten zum Ausdruck kam.  Nichtsdestotrotz
	existieren auch wissenschaftliche Arbeiten über Teilbereiche des
	Contiki NetStacks.  Über das \emph{ContikiMAC} Radio Duty Cycling
	Protokoll findet sich beispielsweise viel Material
	\autocite{dunkels11contikimac, ko12pragmatic, duquennoy11lossy}.  Aber
	auch allgemein zur Kommunikation zwischen Sensorknoten wird viel
	geforscht \autocite{lunden11politecast, dunkels11announcement,
	durvy08making}.


\section{Anforderungen an den \acl{NetStack} in Contiki}

	Da Contiki ein internetfähiges Betriebssystem darstellen soll, benötigt
	es einen \acl{NetStack}.  Ferner ist Contiki für verschiedene
	eingebettete Systeme mit unterschiedlichsten Hardware-Ausstattungen
	gedacht.  Das eine eingebettete System möchte lediglich einen
	IPv4-Stack benutzen, das andere wiederum einen IPv6-Stack, ein weiteres
	soll lediglich \emph{Mesh-under-Routing}-Funktionalität bieten.

	Hierzu bietet Contiki einen \acf{NetStack}, der sich flexibel auf
	verschiedene Bedürfnisse anpassen lässt. Dennoch soll die Flexibilität
	nicht all zu groß sein, sonst wirkt sich dieses negativ auf die
	Überschaubarkeit aus.

	Damit dieses Kapitel nicht den Rahmen sprengt, wird lediglich
	der \acl{NetStack} behandelt, nicht aber die \acs{uIP}-Implementation
	oder die Routing-Protokolle.


\section{Grundlagen der drahtlosen Kommunikation in Sensornetzen}

	Mehrere drahtlos kommunizierende Sensorknoten bilden ein drahtloses
	Sensornetz (\engl{wireless sensor network}).  Dabei formen die
	Sensorknoten ein infrastruktur-basiertes oder sich selbst
	organisierendes Ad-hoc-Netz.  Diese Ad-hoc-Netze sind vermascht
	(\engl{mesh}), \dhx jeder Knoten ist mit einem oder mehreren anderen
	Knoten verbunden. Zu übertragene Informationen werden über
	Sensorknoten, welche Routing-Funktionalitäten (Routing-Knoten)
	besitzen, weitergereicht, bis diese Informationen ihr Ziel erreicht
	haben.  Es findet somit eine \emph{Multi-Hop-Kommunikation} statt.  Oft
	wird ein Sensorknoten zum Coordinator des Sensornetzes ernannt, der
	dann die Aufgabe hat, das Übertragungsmedium zu organisieren.

	Da es sehr viele Knoten in einem solchen Netz gibt und Anzahl sowie
	Standorte dieser Knoten variieren können, muss man von einem
	unvorhersagbaren und dynamischen Verhalten der Knoten ausgehen. Zum
	einen können neue Knoten während des Betriebes hinzukommen, andere
	können ausfallen.  Damit ist die Netztopologie nicht sicher, wodurch
	Netzprotokolle für Sensornetze genau diese Eigenschaft berücksichtigen
	müssen.

	Aus diesem Grund wird zur Zeit im Bereich Sensornetze zu den
	Netzprotokollen sehr viel geforscht.

\subsection{Aufgaben der Netzprotokolle}
\label{sec:AufgabenNetzprotokolle}

	Zusammen mit den Eigenschaften der Sensorknoten muss eine Kommunikation
	geschaffen werden, die mindestens folgende Aufgaben erfüllt:

	\begin{itemize}
	\item 	Daten so effizient wie möglich übermitteln
	\item 	Energiereserven der Sensorknoten schonen, \dhx
		(batteriebetriebene) Knoten sollten möglichst lange
		Schlafzeiten haben
	\item 	energieintensive Bauteile wie den Funksensor so oft wie
		möglich ausschalten
	\item 	bei Ausfall von Routing-Knoten Alternativwege verwenden,
		sofern vorhanden
	\item 	Daten vorhalten, bis Ziel erreichbar ist
	\item 	klassische Kommunikationsprobleme in Rechnernetzen
		auf ein Minimum reduzieren
	\end{itemize}

	Da man nicht allen Aufgaben gerecht werden kann, gilt es Netzprotokolle
	zu finden, die ein Optimum für die jeweiligen Systemanforderungen
	darstellen.  Beispielsweise besitzt ein Sensornetz zur
	Hausautomatisierung andere Systemanforderungen als ein Sensornetz, das
	die Kommunikation innerhalb eines Roboterschwarms sicher stellt.

	Das Verhalten der Netzknoten lässt sich dabei in vier Schritte
	gliedern: Initialisierung, Tagesablauf, Kommunikationsschema sowie
	Routing.

	\begin{description}

	\item[Initialisierung]
		Neue Netzknoten integrieren sich in das Sensornetz, indem
		sie ihre Nachbarn detektieren. Die Netztopologie wird verändert.
		Hierbei ist ein Verfahren gefordert, welches ein erfolgreiches Routing
		verspricht.

	\item[Tagesablauf]
		Der Wach-Schlaf-Zyklus (\engl{duty cycle}) bestimmt
		den Tagesablauf eines (batteriebetriebenen) Netzknotens.
		Das Ziel des \emph{Duty-Cycling} ist es,
		möglichst lange Schlafzeiten zu erreichen.
		Wenn benachbarte Knoten eines Senders beteiligt werden,
		obwohl sie nicht Empfänger der Nachricht sind,
		spricht man von \emph{Overhearing}.
		Overhearing ist in Sensornetzen zu vermeiden, weil das Mithören
		Zeit und Energie kostet. Es gibt in der Forschung Bestrebungen,
		Duty-Cycling und Overhearing sinnvoll zu verbinden:
		\citetitle{Thesis:VereinbarkeitRDCOverhearing} \autocite{Thesis:VereinbarkeitRDCOverhearing}.

	\item[Kommunikationsschema]
		Dieses Schema beschreibt, wie ein einzelner Datenaustausch zwischen
		zwei Sensorknoten abläuft, wobei die Übertragungsgeschwindigkeit
		und auch Fehlerfreiheit wichtige Kriterien sind. Zusätzlich dürfen sich
		die Netzknoten nicht gegenseitig stören.

	\item[Routing]
		Im Gegensatz zum Kommunikationsschema beschreibt Routing den Pfad durch das
		Sensornetz vom Sender zum Empfänger. In einem vermaschten Netz können dazu
		mehrere Pfade existieren. Das Ziel des Routings sollte vor allem die
		ausgewogene Netzauslastung sein, \dhx es muss nicht immer der kürzeste Pfad
		gewählt werden, sonst kann es zur einseitigen Netzbelastung und sogar zum
		Ausfall wichtiger Netzknoten kommen.

	\end{description}

\subsection{Unterschied zu herkömmlichen Netzprotokollen}

	Untersuchungen haben gezeigt, dass herkömmliche Netzprotokolle wie
	\emph{IEEE 802.11} oder \acs{CSMA} die Lebensdauer der Netzknoten stark
	herabsenkt:

	In herkömmlichen Netzprotokollen wird erwartet, dass ein Empfänger
	ständig empfangsbereit ist. \emph{Nutzloses Mithören} (\engl{idle
	listening}) am Medium verschwendet viel Energie, weil der Sensorknoten
	während dieser Zeit nicht schlafen kann.  Wenn eine \emph{Kollision}
	auftritt und damit das Paket erneut übertragen wird, steigt der
	Energieverbrauch und auch die Latenz, insbesondere, wenn es sich um ein
	großes Paket handelt.  Auch ist der \emph{Overhead durch
	Kontrollpakete} gering zu halten, da Kontrollpakete keine Sensordaten
	enthalten und alle Sensorknoten ein gemeinsames Primärziel verfolgen.

	Aufgrund der Anforderung, dass ein Sensornetz ein Gesamtziel verfolgt,
	hebt es sich zusätzlich von anderen Ad-hoc-Netzen ab.  Es existieren
	somit nicht viele verschiedene Anwendungen, die um die gemeinsame
	Ressource, den \emph{Übertragungskanal}, konkurrieren, sondern nur eine
	einzige. Anders formuliert bedeutet dieses, dass die
	Sensornetz-Anwendung quasi mit sich selbst konkurriert.  Damit besteht
	nur ein geringes Bestreben, Fairness zwischen allen Netzknoten zu
	erreichen.


	% MAC und Low-Power-MAC bzw. RDC
	% Routing

	% 802.15.4
	% 6LoWPAN
	% ??

\section{Konzept des Contiki NetStacks}

	Da ein eingebettetes Betriebssystem wie Contiki-OS auf
	unterschiedlichsten Sensor-Platformen sowie unterschiedlichsten
	Bedürfnissen an den \acf{NetStack} zum Einsatz kommen soll, wurde auf
	ein flexibles \acl{NetStack}-System Wert gelegt.

	Vor allem die unteren, transportorientierten Schichten des
	OSI-Schichtenmodells sind entscheidend, um das Sensorsystem optimal an
	die Systemanforderungen anzupassen.  Diese transportorientierten
	Schichten lösen die Aufgaben des Netzprotokolls, die in
	\autoref{sec:AufgabenNetzprotokolle} beschrieben sind.

	\begin{description}
	\item[Bitübertragungsschicht]
		Die physikalische Schicht (Bitübertragungsschicht)
		steuert den Funksensor (Radio) an
		und nennt sich \emph{Radio~Driver}.

	\item[Sicherungsschicht]
		Über der physikalischen Schicht befindet sich
		die \emph{Data~Link}-Schicht (Sicherungsschicht).
		Beide Schichten steuern den Zugriff auf das Übertragungsmedium.

		Die zu versendenen und empfangenen Nachrichten werden häufig in
		Frames gepackt, welche Größe und weitere Eigenschaften der
		Nachricht in einem Header speichern. Hierzu kommt in Contiki
		der Framer zum Einsatz.

		Die Sicherungsschicht besteht wiederum aus den Schichten
		\ac{LLC} und \ac{MAC}:
		\begin{description}
		\item[\acf{MAC}]
			Da in eingebetteten Systemen die Energieeffizienz
			besonders wichtig ist, wurde das \ac{MAC}-Protokoll in
			zwei Ebenen eingeteilt:

			Der untere Layer ist für den \acf{RDC} verantwortlich
			und sorgt damit für das Sparen der Energie.

			Der obere ist für die Steuerung des Zugriffs
			auf das Übertragungsmedium zuständig.

		\item[\acf{LLC}]
			Für die \acs{LLC}-Schicht ist in Contiki kein separate
			Ebene bereitgestellt worden.

			Einige Netzwerk-Stacks wie das ZigBee Protocol
			benötigen die \acs{LLC}-Schicht nicht, weil eine
			leichtgewichtige Netzwerkarchitektur gefordert war.

			Das \acf{SP} nutzt einen \acs{SP}-Adaption Sublayer,
			der sich im Prinzip auf der selben Höhe der
			\acs{LLC}-Schicht befindet. Dadrüber befindet sich das
			\acl{SP}, welches als eine \emph{Bridge} angesehen
			werden kann, um eine Koexistenz mehrerer
			Netzwerk-Protokolle zu erreichen.
			\autocite{he2007implSPforContiki}
		\end{description}

	\item[Vermittlungsschicht]
		Die dritte OSI-Schicht ist die Netzwerk-Schicht
		(Vermittlungsschicht) und stellt auch in Contiki
		netzübergriefende Adressen bereit und organisiert das Routing
		zwischen Netzknoten.
	\end{description}

	Zusammenfassend gliedert sich die Implementation des \emph{Contiki
	NetStacks} in folgende fünf Schichten (\engl{layer}): Radio Driver,
	Framer, RDC Driver, MAC Driver und Network Driver.  Ein Vergleich
	zwischen dem OSI-Referenzmodell und den Contiki NetStack Schichten ist
	in \autoref{fig:ContikiNetStackKonzept} zu sehen.

\begin{figure}
\centering
\begin{tikzpicture}[
	>=stealth',
	NS Schicht/.style={
		top color=white,
		bottom color=blue!50!black!20,
		draw=blue!50!black!50,
		minimum width=3.5cm,
		rectangle,
		rounded corners=1mm
		},
	OSI Schicht/.style={
		top color=white,
		bottom color=orange!50!black!20,
		draw=orange!50!black!50,
		minimum width=8mm,
		rectangle,
		rounded corners=1mm
		},
	]

\node [NS Schicht] (radio) {Radio Driver};
\node [NS Schicht] (framer) [above of=radio] {Framer};
\node [NS Schicht] (rdc) [above of=framer,yshift=.2cm] {RDC Driver};
\node [NS Schicht] (mac) [above of=rdc] {MAC Driver};
\node [NS Schicht] (nw) [above of=mac,yshift=.2cm] {Network Driver};

\node (s0h) at (framer) [xshift=-7cm] {};
\node (s1h) at (mac) [xshift=-7cm] {};
\node (s2h) at (nw) [xshift=-7cm] {};

% Boxen für die OSI-Schichten
\begin{pgfonlayer}{background}
	\node [OSI Schicht] (s0) [fit=(framer) (radio) (s0h)] {};
	\node [OSI Schicht] (s1) [fit=(mac) (rdc) (s1h)] {};
	\node [OSI Schicht] (s2) [fit=(nw) (s2h)] {};
\end{pgfonlayer}

% Label für die OSI-Schichten
\node [anchor=north west] at (s0.north west) {1 Bitübertragung (Physical)};
\node [anchor=north west] at (s1.north west) {2 Sicherung (Data Link)};
\node [anchor=north west] at (s2.north west) {3 Vermittlung (Network)};

% Pfeile rechts

\draw [->] ([xshift=1cm]radio.east) to node [sloped,above] {eingehende Nachricht} ([xshift=1cm]nw.east);
\draw [<-] ([xshift=2cm]radio.east) to node [sloped,above] {ausgehende Nachricht} ([xshift=2cm]nw.east);

\end{tikzpicture}
\captionbelow{Contiki NetStack Konzept}
\label{fig:ContikiNetStackKonzept}
\end{figure}


	Da der \acl{NetStack} in fünf voneinander losgelösten Schichten
	konzipiert ist, die über vorgegebenen APIs miteinander kommunizieren.
	Dadurch ist die Implementation einer Schicht von den anderen Schichten
	unabhängig austauschbar und der \emph{Contiki NetStack} flexibel auf
	unterschiedliche Anwendungen anpassbar.

	Alle über der Netzwerk-Schicht liegenden Schichten sind nicht direkt
	Teil des \acl{NetStack}s, da sich ihre Implementationen sehr stark
	unterscheiden und damit keine sinnvolle API gebildet werden kann.

	Bei der Erstellung des anwendungsspezifischen \acl{NetStack}s müssen
	die verwendeten Protokolle auf Kompatiblität zueinander geprüft werden.
	Auch ist es sinnvoll, alle Netzknoten eines Sensornetzes mit dem
	gleichen \acl{NetStack} auszustatten.

\section{Funktionsweise}

	In diesem Abschnitt wird jeder der fünf Contiki Netzwerkschichten und
	ihre Aufgaben beschrieben. Ziel ist es hierbei, ein Verständnis für den
	Aufgabenumfang eines Treibers im Contiki NetStack zu entwickeln.

	Im folgenden \autoref{sec:NetStack:SendenEmpfangen} wird dann am
	Beispiel eines konkreten Contiki NetStacks das Empfangen und Versenden
	einer Nachricht demonstriert.

\subsection{Radio Driver}
\label{sec:NS:Funktionsweise:RadioDriver}

	Der Radio Driver steuert den Funksensor an und ist im Allgemeinen für
	das Senden und Empfangen von Nachrichten zuständig.

	Der Funksensor ist dabei entweder direkt im Mikrocontroller integriert
	(1-Chip-Lösung) oder über ein allgemeines Interface am Mikrocontroller
	angeschlossen (2-Chip-Lösung).  Daher ist ein Radio Driver keine CPU-
	oder platformunabhängige Implementation.

	Die meisten Hersteller von Funksensoren implementieren einen größeren
	Funktionsumfang, als es nötig wäre. So können bestimmte
	Funk-Funktionalitäten, wie das automatische Senden eines Ack-Signal zur
	Mitteilung des Erhalts eines Pakets, an die Hardware übergeben werden.
	So muss dieses nicht in der Software gelöst werden, was zur Folge hat,
	dass Funksensoren \iA für bestimmte Standards ausgelegt sind:

	Der Funksensor \emph{AT86RF230} ist beispielsweise besonders für die
	Standards \ieeeframe, ZigBee, 6LoWPAN (und weitere) geeignet.  Er
	besitzt einen besonderen \emph{\ieeeframe-2003 Hardware Support} und
	erleichert die Software um folgende Funktionalitäten:

	\begin{itemize}
	\item 	\acf{CCA},
	\item 	Energy Detection,
	\item 	Berechnung der \acl{FCS} (\acs{FCS}, \acs{CRC}),
	\item 	automatisches \acs{CSMA/CA},
	\item 	automatische, erneute Übertragung eines Frames,
	\item 	automatisches Bestätigen des Erhalts eines Frames (ACK-Signal),
	\item 	automatisches Filtern von Adressen.
	\end{itemize}

	Das Umsetzen solcher Funktionalitäten auf der Hardwareebene erfordert
	einen besonderen Umgang im Netzwerk-Stack.  Höhere Schichten müssen
	herausfinden, ob diese Funktionalitäten auf der Hardware umgesetzt
	werde können.  In C können hierzu einfach Makros verwendet werden,
	welche con Contiki auch genutzt werden, um einzelne Hardwarefunktionen
	zu aktivieren.

\subsection{Framer}

	Bei eingehenden Nachrichten hat der Framer die Aufgabe, den Payload der
	Nachricht vom Frame zu separieren sowie die Informatonen, die im Frame
	stecken, lesbar in einem Nachrichten-Puffer für die Netzwerk-Schichten
	aufzubereiten.

	Im Framer selbst werden diese Informationen nicht verarbeitet.  Dies
	geschieht dann in den jeweilig zuständigen Schichten.

	Bei ausgehenden Nachrichten werden die Informationen, die die
	Netzwerk-Schichten im Nachrichten-Puffer vorbereiet haben, in ein Frame
	verpackt, so dass der Framer des Empfängers diese Informationen lesen
	kann.  Daher ist es wichtig, auf allen Netzknoten denselben Framer zu
	verwenden.

	Deswegen wurde in Bezug auf Sensornetze der Standard \ieeeframe{}
	entwickelt, um einen einheitlichen Nachrichtenaustausch (auf den
	unteren zwei Netzwerk-Schichten) sicherzustellen.  Da \ieeeframe{}
	sowohl auf der Bitübertragungsschicht sowie auf der Sicherungsschicht
	arbeitet, muss der Framer eng mit der \acs{MAC}-Schicht sowie der
	\acs{RDC}-Schicht zusammenarbeiten.

\subsection{RDC Driver}
	Der Radio Duty Cycling Driver kümmert sich darum, dass der Funksensor
	\idR nicht länger als nötig in Betrieb ist.  Der Funksensor wird also
	so oft wie möglich aus- und zu entsprechenden Zeiten wieder
	eingeschaltet.

	Um einen energiesparsamen Betrieb zu gewährleisten und um regelmäßig
	emfangsbereit zu sein, gibt es in der Forschung viele Protokollansätze,
	die unter dem Begriff Radio Duty Cycling Protokolle oder auch Low Power
	MAC Protokolle fassen kann.  Einige dieser Ansätze sind nur in
	bestimmten Sensornetzanwendungen sinnvoll, weshalb es in Contiki für
	diese Funktionalität eine extra Netzwerk-Schicht, die RDC-Schicht,
	gibt.

	Der RDC Driver liegt folglich zwischen dem Radio Driver und dem MAC
	Driver. Weil der Radio Driver hardwareabhängig implementiert ist und
	der RDC Driver hardwareunabhängig, ist es sinnvoll, dass der RDC Driver
	den Framer anweist, das Nachrichtenpaket zu parsen oder in eine Frame
	zu verpacken.

	In Contiki gibt es für den RDC Driver verschiedene implementierte
	Protokolle: Eines dieser so genannten Low Power MAC-Protokolle ist
	ContikiMAC, welches \iA in Contiki als Standardpotokoll verwendet wird.
	Es befindet sich in Contiki aber auch ein Protokoll, das das \acf{LPP}
	umsetzt, sowie das weiter verbreitete X-MAC und die kompaktere Variante
	CX-MAC.

	Im Gegensatz zu den Low Power MAC-Protokollen gibt es in Contiki auch
	Implementierungen, die keinen Radio Duty Cycle verwenden: In diesem
	Bereich ist die NullRDC Implementierung die aktuellste.  Gleichfalls
	existiert eine Variante, welche weder einen RDC verwendet noch einen
	Framer einsetzt: der NullRDC NoFramer.

	Bei diesen (einfachen) Protokollen bleibt der Funksensor \iA dauerhaft
	eingeschaltet, wodurch kein energieeffizientes Arbeiten sichergestellt
	ist.  Deshalb sind diese nicht auf Energieeffizienz bedachten
	Protokolle nur bei einem \ac{FFD} einzusetzen, da sie \idR nicht
	batteriebetrieben sind. Ein \ac{RFD} unterscheidet sich dadurch vom
	\ac{FFD}, dass es nicht den vollen standardisierten Funktionsumfang
	bieten muss, wodurch allerdings keine Kommunikation mit einem anderen
	\ac{RFD} möglich ist.

	Allgemein gewinnen in der Informatik energiebewusste Technologien immer
	mehr an Bedeutung, wodurch ein Radio Duty Cycle auch bei einem
	\acl{FFD} eingesetzt werden sollte.

\subsection{MAC Driver}
	Der MAC Driver setzt die Anforderungen der MAC-Schicht um, \dhx er muss
	sich um die Organisation des Übertragungsmediums kümmern.  Hierbei ist
	es sinnvoll die Funktionalitäten des Radio Drivers einzubeziehen, um
	bereits in Hardware implementierte Funktionen zu verwenden.

	In Sensornetzen mit batteriebetriebenen Netzknoten wird es immer wieder
	vorkommen, dass Netzknoten nicht erreichbar sind, weil sie gerade
	schlafen.  Diese Schlafzyklen werden schließlich vom RDC Driver
	bestimmt, so dass dieser der MAC-Schicht Informationen zukommen lassen
	kann, wie die eigenen Schlafzyklen sich verhalten.

	Aber er kann auch schätzen, wie sich das Schlafverhalten der anderen
	Netzteilnehmer verhält, da man \idR davon ausgehen kann, dass alle
	Netzteilnehmer das gleiche RDC-Protokoll verwenden.  Hieraus kann der
	MAC Driver bestimmen, wann es sinnvoll ist -- unter Einhaltung der
	Funk-Zeiten des MAC-Protokolls --, dem Empfänger die Nachricht zu
	schicken.

	So kann in Zusammenarbeit mit dem RDC Driver, die Funkkommunikation auf
	ein Mindestmaß reduziert werden.

	Auf der anderen Seite arbeitet der MAC Driver mit dem Network Driver
	zusammen: Dieser nimmt zum einen von Network Driver zu versendene
	Nachrichtenpakete entgegen und leitet sie zu bestimmten Zeiten an den
	RDC Driver weiter.  Zum anderen werden empfangene Nachrichten an den
	Network Driver weitergeleitet, sofern die empfangenen Nachrichten für
	den eigenen Netzknoten bestimmt sind.

\subsection{Network Driver}

	Der Netzwerk Driver stellt die Funktionalität der Vermittlungsschicht
	bereit. Hierzu zählt insbesondere das Bereitstellen von
	netzwerkübergreifender Adressen, das Routing und Verwalten von
	Routingtabellen -- sofern gewünscht -- sowie allgemein der Aufbau von
	Paketen.

	Da Nachrichten länger als die im Framer festgelegte, maximale Breite
	sein können, müssen diese Nachrichten in Fragmente unterteilt werden.
	Das Einteilen in Fragmente, sofern erforderlich, sowie das
	Zusammenfügen einer Reihe von empfangener Fragmente übernimmt der
	Network Driver.

	Hierzu bietet sich in Contiki der weit verbreitete Standard
	\acs{6LoWPAN}.  Eine Alternative ist der RIME Stack.  der einen Satz
	leichtgewichtiger Kommunikationsprimitiven bietet: von
	\emph{best-effort anonymous local area broadcast} bis hin zu
	\emph{reliable network flooding} \autocite{dunkels07adaptive}.


\begin{table}[t]
\begin{threeparttable}[c]
\centering
\caption{Überblick zu den implementierten Treibern in Contiki 2.6}
	\label{tab:ContikiNetStackDriver}
\begin{tabular}{lllll}
	\toprule
	  \multicolumn{1}{c}{{\theadhll{Network}}}
	  & \multicolumn{1}{c}{{\theadhll{MAC}}}
	  & \multicolumn{1}{c}{{\theadhll{RDC}}}
	  & \multicolumn{1}{c}{{\theadhll{Framer}}}
	  & \multicolumn{1}{c}{{\theadhll{Radio}}}	\tabularnewline
	\midrule
	SicsLoWPAN	& CSMA		& ContikiMAC		& \ieeeframe	& AVR rf230 (bb)\tnote{2}	\tabularnewline
	RIME		& TDMA\tnote{1}		& X-MAC				& NullMAC	& TI cc2430\,rf				\tabularnewline
				& CTDMA\tnote{1}	& CX-MAC			& 			& TI cc2530\,rf				\tabularnewline
				& RIME UDP			& LPP 				& 			& ARM stm32w				\tabularnewline
				& NullMAC			& NullRDC			& 			& 							\tabularnewline
				& 					& NullRDC NoFramer	& 			& 							\tabularnewline
	\bottomrule
\end{tabular}
\begin{tablenotes}\footnotesize
\item Die hier vorgestellten Implementationen wurden nicht auf Vollständigkeit getestet.
\item[1] \acs{TDMA} und \acs{CTDMA} wurden seit Version 2.6.x aus Contiki entfernt \cite{Contiki:GitHub}.
\item[2] rf\,230\_bb ist die überarbeitete Version des Funkmoduls rf\,230.
\end{tablenotes}
\end{threeparttable}
\end{table}

%\begin{table*}[t]
	\centering
	\caption{Überblick zu den implementierten Treibern in Contiki 2.6}
	\label{tab:ContikiNetStackDriver}
	\begin{subtable}[l]{0.3\textwidth}
		\centering
		\begin{tabular}{l}
			\toprule
			\midrule
				SicsLoWPAN \tabularnewline
				RIME \tabularnewline
			\bottomrule
		\end{tabular}
		\caption{Network driver}
	\end{subtable}
	\hfill
	\begin{subtable}[l]{0.3\textwidth}
		\centering
		\begin{tabular}{l}
			\toprule
			\midrule
				CSMA \tabularnewline
				TDMA \tabularnewline
				CTDMA \tabularnewline
				RIME UDP \tabularnewline
				NullMAC \tabularnewline
			\bottomrule
		\end{tabular}
		\caption{MAC driver}
	\end{subtable}
	\hfill
	\begin{subtable}[l]{0.3\textwidth}
		\centering
		\begin{tabular}{l}
			\toprule
			\midrule
				ContikiMAC \tabularnewline
				X-MAC \tabularnewline
				CX-MAC \tabularnewline
				LPP \tabularnewline
				NullRDC \tabularnewline
				NullRDC NoFramer \tabularnewline
			\bottomrule
		\end{tabular}
		\caption{RDC driver}
	\end{subtable}
	\hfill
	\begin{subtable}[l]{0.3\textwidth}
		\centering
		\begin{tabular}{l}
			\toprule
			\midrule
				Framer 802.15.4 \tabularnewline
				Framer NullMAC \tabularnewline
			\bottomrule
		\end{tabular}
		\caption{Framer}
	\end{subtable}
	\hfill
	\begin{subtable}[l]{0.3\textwidth}
		\centering
		\begin{tabular}{rl}
			\toprule
				\multicolumn{1}{c}{{\theadhll{CPU}}}
				& \multicolumn{1}{c}{{\theadhll{Radio}}} \tabularnewline
			\midrule
				AVR & rf230 \tabularnewline
				cc2430 & cc2430\_rf \tabularnewline
				cc253x & cc2530\_rf \tabularnewline
				stm32w108 & stm32w \tabularnewline
				mc1322x & Contiki MACA \tabularnewline
			\bottomrule
		\end{tabular}
		\caption{Radio driver}
	\end{subtable}
\end{table*}



\section{Senden und Empfangen von Nachrichten}
\label{sec:NetStack:SendenEmpfangen}

	Mit der Entwicklung von IPv6 ist es möglich geworden, jedem Gerät
	(\emph{Ding}\,) eine eigene Internet-Adresse zu geben.  Das
	\emph{Internet der Dinge} (\engl{Internet of Things}) ermöglicht also,
	das komplette Sensornetz mit dem Internet zu verbinden.  Dieses
	Sensornetz wird dann zum \emph{Wireless Embedded Internet}.

	Damit die Anbindung möglichst transparent ist, wurde versucht, IPv6 auf
	Sensorknoten zu implementieren.  Da aber das komplette Internet
	Protokoll sehr umfangreich ist, wäre der Energieverbrauch und auch der
	Speicherplatzbedarf der Implementierung drastisch gestiegen.  Daher
	wurde eine leichtgewichtige Variante des IPv6 entwickelt:
	\acf{6LoWPAN}.

	Da dieses Protokoll sich weitgehend durchgesetzt hat, wird das Senden
	und Empfangen von Nachrichten anhand der \acs{6LoWPAN}-Implementierung
	erläutert.  Um genauer zu sein, wird der in
	\autoref{tab:ContikiNetStackExample} gezeigte \acl{NetStack}
	betrachtet.
\begin{table}[t]
\begin{threeparttable}[c]
\centering
\caption[Contiki NetStack Beispiel]{Contiki NetStack Beispiel.
	Zu sehen ist der Vergleich zwischen dem 6LoWPAN-Stack und die Implementation in Contiki.}
	\label{tab:ContikiNetStackExample}
\begin{tabular}{rrlrl}
	\toprule
	\multicolumn{3}{c}{\theadhll{6LoWPAN NetStack}}
	& \multicolumn{2}{c}{\theadhll{Contiki NetStack}}
	\tabularnewline
	\cmidrule(r){1-3}
	\cmidrule(l){4-5}
	  \multicolumn{2}{c}{\theadhll{OSI-Schicht}}
	  & \multicolumn{1}{c}{{\theadhll{Protokoll}}}
	  & \multicolumn{1}{c}{{\theadhll{Contiki-Schicht}}}
	  & \multicolumn{1}{c}{{\theadhll{Implementation}}}
	  \tabularnewline
	\midrule
	4	& Transport 	& UDP					& 					& uIP-UDP 				\tabularnewline
	\addlinespace
	3	& Vermittlung 	& IPv6 / ICMPv6			& Network Driver	& SicsLoWPAN (uIPv6) 	\tabularnewline
	-	& Adaptation\tnote{1}	& 6LoWPAN Adaptation	& 					& 					\tabularnewline
	\addlinespace
	2	& Sicherung 	& \ieeeframe{} MAC		& MAC Driver 		& CSMA 				\tabularnewline
		& 				&						& RDC Driver 		& X-MAC\tnote{2} 			\tabularnewline
	\addlinespace
	1	& Bitübertragung & \ieeeframe{} PHY	& Framer 			& Framer\,802.15.4 	\tabularnewline
		& 				 &						& Radio Driver 		& rf\,230\_bb 		\tabularnewline
	\bottomrule
\end{tabular}
\begin{tablenotes}\footnotesize
\item[1] 6LoWPAN fügt in das OSI-Referenzmodell eine neue Schicht ein: der Adaptation Layer.
\item[2] 6LoWPAN definiert keinen genauen \ac{RDC}. In Contiki wird oft ContikiMAC als Default verwendet.
\end{tablenotes}
\end{threeparttable}
\end{table}



\subsection{Empfangen einer 6LoWPAN-Nachricht}

\minisec{Radio Driver}

	Jeder Empfang einer Nachricht wird mit einem Interrupt beim
	AVR-Mikrocontroller eingeleitet. Dadurch weiß der Radio Treiber
	rf230~bb, dass im Nachrichten-Puffer des Funksensors die Nachricht zum
	Lesen bereitliegt.

	In Contiki ist es sinnvoll, hierfür einen Prozess zu erstellen, der
	sich um den Empfang und die weitere Verarbeitung der Nachricht kümmert.
	Die \ac{ISR} pollt diesen Prozess, wodurch der Interrupt möglichst
	schnell verlassen wird und die CPU für andere wichtige Aufgaben wieder
	zur Verfügung steht.

	Sobald der Empfangsprozess an der Reihe ist, wird der Paket-Puffer
	geleert und das Paket aus dem Nachrichten-Puffer des Funksensors in den
	Paket-Puffer gelesen.

	Die \lstinline=read=-Operation ist ein kritischer Abschnitt, in welchem
	Interrupts kurzfristig deaktiviert werden müssen.  Damit wird
	verhindert, dass die \ac{ISR} nicht den gleichen Puffer beschreibt, von
	dem gerade gelesen werden soll.

	Nun wird die \ac{RDC}-Schicht über die Ankunft einer neuen Nachricht
	mittels \lstinline=input=-Callbacks informiert.


\minisec{RDC Driver}

	Nach \autoref{tab:ContikiNetStackExample} kommt nun die
	X-MAC-Implementation zum Einsatz. Der \lstinline=input=-Callback ruft also
	die Funktion \lstinline=input_packet= auf.

	Diese Funktion muss zuerst die empfangene Nachricht parsen, weshalb nun
	der Framer zum Einsatz kommt.

	Wenn man beispielsweise den RDC Driver \emph{NullRDC NoFramer}
	verwendet, dann erwartet der Empfänger, dass die Nachricht -- wie der
	Name besagt -- in keinem Frame-Format vorliegt.

\medskip

	Nach dem Parsen der Nachricht durch den Framer untersucht der Empfänger
	mit Hilfe eines Packet Dispatchers, um welche Art von Paket es sich
	handelt:

	Im Falle von X-MAC kann dies ein Strobe-, ein Strobe-ACK-, ein
	Announcement- oder Data-Paket sein: Letzteres beinhaltet die
	eigentliche Nachricht und die ersten drei koordinieren den
	Datenaustausch zwischen zwei Sensorknoten.

	Wann welches Paket verschickt wird und wie der Schalf-Rhythmus
	aussieht, wird im Kommunikationsschema festgelegt.

	Wenn der Paket-Typ festgestellt wurde, wird bei einigen Paket-Typen die
	Zieladresse überprüft. Wenn diese nicht mit unserer übereinstimmt oder
	die Nachricht kein Broadcast war, so wird das Paket ebenfalls
	verworfen.

	\begin{description}
	\item[Data-Paket]
		Bei einem Data-Paket wird die Sequenznummer mit den letzten
		paar empfangenen Paketen verglichen, damit doppelte Pakete
		sofort verworfen werden können. Außerdem wird der Funksensor
		deaktiviert, um Energie zu sparen.  Sobald das Paket
		verarbeitet wurde, kann der gesamte Mikrocontroller sich
		schlafen legen.

	\item[Strobe-Paket]
		Bei einem Strobe-Paket ohne Empfängeradresse wird bei X-MAC
		davon ausgegangen, dass ein Broadcast-Paket folgt, so dass in
		diesem Fall der Funksensor aktiv bleibt bzw. angeschaltet wird.

	\item[Announcement-Paket]
		Da regelmäßige Beacon-Pakete sehr energieaufwändig sind, um ein
		Sensornetz zu organisieren, ist die Idee eines Announcemnt
		Layers \autocite{dunkels11announcement} entstanden.  Hierbei
		werden vor allem mehrere Beacons in einem Announcement-Paket
		gesammelt, um die Anzahl der Übertragungen zu minimieren und
		somit Energie zu sparen.

		Dieses Feature kann auch bei X-MAC in Contiki aktiviert werden,
		ist für diese Betrachtung aber nicht relevant.
	\end{description}

	Da bei RDC Protokollen -- auch \emph{Low power MAC-Protokolle}
	genannt -- \idR davon ausgegangen wird, dass sowohl Empfänger
	als auch Sender das gleiche Protokoll verwenden, ist es
	ausreichend, nur Data-Pakete an den eigentlichen MAC Driver
	weiterzureichen.

	Es ist durchaus denkbar, dass weitere Eigenschaften an höhere
	Schichten weitergegeben werden können -- wie beispielsweise der
	akkumulierte Energieverbrauch beim Senden und Empfangen.


\minisec{Framer}
	Nach \autoref{tab:ContikiNetStackExample} ist der
	\emph{Framer\,802.15.4} verwendet worden.  Der Empfanger erwartet
	folglich, dass die Nachricht im \ieeeframe-Format vorliegt.

	Der Parser stellt fest, ob die PAN-ID in der Nachricht der eigenen
	entspricht oder ob es sich um eine Broadcast-PAN-ID handelt.

	Wenn die Nachricht nicht die eigene oder die Broadcast-PAN-ID enthält,
	wird das Paket verworfen.

	Andernfalls ist die Nachricht entweder für das
	PAN-Netzwerk\footnote{Nach \ieeeframe{} wird die PAN-ID dazu benutzt,
	um das Netzwerk in unterschiedliche Cluster einzuteilen.  Jeder Cluster
	hat seinen eigenen PAN-Coordinator. Daneben gibt es noch einen
	First-PAN-Coordinator, der alle Cluster koordiniert.} bestimmt oder es
	handelt sich um einen Broadcast.  In beiden Fällen wird der
	Paket-Puffer mit den in der Nachricht befindlichen Informationen
	gefüllt: Hierzu zählt die Adresse des Senders und Empfängers, die
	Paket-ID sowie ein Prending-Flag.

	Nun wird die Länge des \ieeeframe-Overheads\footnote{Die Länge des
	\ieeeframe-Overheads errechnet sich durch die Länge der Nachricht minus
	der Länge des Payload.} an den RDC Driver zurückgegeben, damit dieser
	sofort entsprechende Maßnahmen ergreifen kann, falls die Länge des
	Overheads nicht dem Standard entspricht oder ein Fehler aufgetreten
	ist.

\minisec{MAC Driver}
	Beim Empfangen einer Nachricht hat der CSMA MAC Driver nicht viel zu
	tun: Er nimmt lediglich die Information vom RDC Driver entgegen, dass
	eine Nachricht eingegangen ist und leitet diese direkt an die
	Netzwerk-Schicht weiter.

\minisec{Network Driver}
	Der Network Driver \emph{SicsLoWPAN} kann nun das \acs{6LoWPAN}-Paket
	verarbeiten und im Falle einer fragmentierten Nachricht den kompletten
	Nachrichten-Payload wieder zusammensetzen.  Das Fragmentieren und
	Wiederzusammensetzen ist die Aufgabe des \acs{6LoWPAN}-Adaptation
	Layers, welcher damit in den Network Driver integriert ist.

	Sofern ein Prozess einen Callback hinterlegt hat, so wird dieser Prozes
	nun über eine neue, eingehende Nachricht informiert, Danach wird der
	TCP/IP-Prozess benachrichtigt, indem die \lstinline=tcpip_input=-Funktion
	ein \event{PACKET\_INPUT}-Event beim TCP/IP-Prozess erzeugt.

\minisec{Höhere Schichten}
	Die Netzwerk-Schicht hat damit die Kontrolle über das Paket an die
	Transport-Schicht abgegeben, welche wiederum die jeweiligen
	Anwendungsschichten informiert.

	Beispielsweise könnte ein Sender dem Empfänger eine Nachricht mittels
	\acs{UDP} an den Port der \acs{CoAP}-Anwendung geschickt haben.


\subsection{Versenden einer 6LoWPAN-Nachricht}

\minisec{Höhere Schichten}
	Angenommen wir haben eine Anwendung, die das \acf{CoAP} verwendet.
	Dann möchte diese Anwendung mittels UDP eine Nachricht versenden.

	Mittels der Funktion \lstinline=uip_udp_packet_send= wird eine
	CoAP-Nachricht an einen bestimmten Port eines anderen Netzknoten
	gesendet.  Wenn IPv6 aktiviert ist, dann wird
	\lstinline=tcpip_ipv6_output= aufgerufen, welche die Nachricht an den
	Network Driver über einen Callback weitergibt.

	Dieser Callback wird beim Initialisieren des Network Drivers übergeben:
	Die TCP/IP-Schnittstelle definiert den Callback \lstinline=outputfunc=,
	welcher mit der Funktion \lstinline=output= des Network Drivers
	\acs{6LoWPAN} verbunden ist.

\minisec{Network Driver}
	Der \acs{6LoWPAN}-Network Driver nimmt mit Aufruf der Funktion
	\lstinline=output= das IP-Paket entgegen und formatiert es so, dass es über
	ein \ieeeframe-Netzwerk versendet werden kann, wobei das IP-Paket in
	Fragmente aufgeteilt wird, sofern dies nötig ist.

	Jedes zu versendene \acs{6LoWPAN}-Paket wird in den Paket-Puffer
	geschrieben und mittels der Funtkion \lstinline=send_packet= versendet,
	wobei der MAC Driver über das zu versendene Paket informiert wird.  Mit
	dem Versand wird ebenfalls der Callback \lstinline=packet_sent= übergeben,
	damit der Network Driver Bescheid bekommt, ob die Nachricht erfolgreich
	versendet wurde oder nicht.

\minisec{MAC Driver}
	Der \acs{CSMA}-MAC Driver nimmt die Nachricht mit dem gleichnamigen
	Aufruf \lstinline=send_packet= entgegen.

	Das \acs{CSMA/CA}-Verfahren hat die Aufgabe, die Belegung des Funkkanal
	zu überwachen und so mögliche Kollisionen vorzubeugen.

	Sobald eine Nachricht versendet werden soll, soll laut dem
	\ieeeframe-Standard das \acs{CSMA/CA}-Verfahren bereits vor dem ersten
	Kanaltest eine zufällige Zeit (\emph{Random-Backoff}\,)
	gewartet werden.  Sofern der Kanal frei ist,
	kann die Nachricht versendet werden, ansonsten muss wieder gewartet
	werden.

	Im Falle eines Broadcast-Pakets wird in der Implementation das Paket
	einfach versendet und damit mittels \lstinline=NETSTACK_RDC.send= an den
	RDC Driver übergeben.

	Das direkte Übergeben des Pakets hat zur Folge, dass die Steuerung der
	Backoff-Zeit in den unteren Schichten erfolgen muss: Das Funkmodul
	\emph{ATmega128RFA1} von AVR implementiert einige der
	\acs{CSMA/CA}-Funktionalitäten hardwareseitig. Damit sind diese
	Funktionalitäten im Radio Driver untergebracht.

	Falls kein Broadcast-Paket versendet werden soll, so wird in der
	\acs{CSMA}-Implementation für den Empfänger eine Paket-Warteschlange
	angelegt, sofern diese noch nicht existiert, und das Paket in diese
	Warteschlange eingereiht. Jeder Empfänger besitzt seine eigene
	Paket-Warteschlange, weshalb es vorkommen kann, dass zum Anlegen des
	Puffers für einen bestimmten Empfänger nicht genügend Speicherplatz
	existiert und deshalb ein Paket verworfen werden muss. Nachdem Pakete
	versendet wurden, wird der Speicherplatz für die entsprechende
	Warteschlange auch wieder freigegeben.

	Falls es sich bei dem hinzugefügten Paket um das erste in der
	Warteschlange handelt, so wird ein Timer mit Ablaufzeit \(0\) gesetzt.
	Dieser Timer ruft bei Ablauf die Funktion \lstinline=transmit_packet_list= mit der
	Warteschlange des Empfängers als Parameter auf.  Diese Funktion
	versendet dann das Paket, indem der RDC Driver mittels
	\lstinline=send_list= informiert wird.

	Hierbei wird zugleich ein MAC-Callback \lstinline=packet_sent= übergeben,
	damit der MAC Driver über das erfolgreiche Versenden informatiert
	werden kann.  Der MAC-Driver übergibt \iA also nicht den Callback des
	Netzwerk-Drivers, da der MAC-Driver selbst weitere Schritte einleiten
	könnte, sofern ein Paket aus irgendeinen Grund nicht den Empfänger
	erreicht hat.

%\todo{erst RDC-Schicht erklären?}
\medskip

	Sobald die Übertragung der Paketliste in der RDC-Schicht beendet wird,
	wird der MAC-Schicht mitgeteilt, ob die Übertragung erfolgreich war.
	Im einfachen Fall einer erfolgreichen Übertragung wird der
	Netzwerk-Schicht diese Information weitergeleitet.  Falls doch ein
	Fehler aufgetreten ist, so können weitere Schritte eingeleitet werden,
	um das Paket erneut zu senden.

	Wie in \autoref{sec:AufgabenNetzprotokolle} erklärt, müssen
	batteriebetriebene Sensorknoten Schlafzeiten einhalten, um eine lange
	Lebensdauer zu garantieren.  Auf Grund dieser Tatsache kann es
	vorkommen, dass der Sender den (batteriebetriebenen) Empfänger nicht
	erreichen konnte, weil jener geschlafen hat.

	Da der Empfänger di Nachricht nicht erfolgreich angenommen hat, wird
	nach \acs{CSMA/CA} eine zufällige Backoff-Zeit gewartet, nach der das
	Paket erneut gesendet werden kann. Dies soll verhindern, dass
	fehlerhafte Übertragungen den Funkverkehr blockieren.

	An dieser Stelle kann die MAC-Schicht einen eigenen, bedingt zufälligen
	Backoff-Timer ermitteln, damit das Schlafverhalten des Empfängers
	berücksichtigt wird. Hierzu wird der im RDC Driver verwendete Radio
	Duty Cycle (\lstinline=NETSTACK_RDC.channel_check_interval=) ausgelesen.
	Sofern der RDC kein Channel-Check-Intervall definiert, benutzt die
	\acs{CSMA}-Implementation die Channel-Check-Rate als
	Berechnungsgrundlage.


\minisec{RDC Driver}
	Mit dem Aufruf von \lstinline=NETSTACK_RDC.send_list= oder
	\lstinline=NETSTACK_RDC.send= wird das Versenden einer Nachricht
	eingeleitet.  Der RDC Driver \emph{X-MAC} verwendet hierzu die interne
	Funktion \lstinline=send_packet=, in welcher ein X-MAC-Header erzeugt und
	dem Payload hinzugefügt wird.  Nun wird der Framer informiert, dass
	eine Nachricht in ein Frame verpackt werden muss.

\medskip

	Nach dem Erzeugen des \ieeeframe-Frames leitet der RDC Driver weitere
	Schritte ein, damit die Nahricht versendet werden kann.

	Je nach Konfiguration des \emph{X-MAC} kann beispielsweise ein
	Broadcast-Strobe-Paket zur Signalisierung einer bevorstehenden
	Broadcast-Nachricht versendet werden.  Gleiches könnte für normale
	Nachrichten verwendet werden, um uninteressierten Empfängern
	mitzuteilen, dass sie die nächste Nachricht nicht empfangen müssen.

	Es können auch verschiedene Optimierungen eingeschaltet sein, die den
	Schlafzyklus der anderen Teilnehmer beobachten und schätzen, womit der
	Energieverbrauch bei der Übertragung verbessert werden kann.

\medskip

	Mit dem Aufruf von \lstinline=NETSTACK_RADIO.send= kann der Versand der
	Nachricht schließlich eingeleitet werden.  Alternativ können auch die
	Funktionen \lstinline=prepare= und \lstinline=transmit=\footnote{Der Vorteil der
	zweiten Methode wird beim Radio Driver erklärt.} des Radio Drivers
	benutzt werden.

\minisec{Framer}
	Der Ablauf des Framers ist im Gegensatz zu den anderen Schichten nicht
	spannend, denn er baut lediglich den \ieeeframe-Frame zusammen:

	Als erstes wird das Frame-Control-Frame (FCF) zusammengesetzt, wobei
	insbesondere die \ieeeframe-Nachricht in der aktuellen Contiki Version
	als Data-Frame gekennzeichnet und die Nachricht als unverschlüsselt
	angegeben wird. An dieser Stelle gilt es zu überlegen, wie man eine
	Verschlüsselung der Nachricht erreichen kann.  \footnote{In
	\autoref{sec:ContikiNetStack:Hinweise} wird dieses Problem kurz
	aufgegriffen.}

	Der implementierte \ieeeframe-Framer unterstützt derzeit nur die
	2003-Version des \ieeeframe-Standards. Dies muss beachtet werden,
	sofern nicht \acs{6LoWPAN} als Network Driver eingesetzt wird.

	Sofern das zu versendene Paket noch nicht versendet wurde, wird die
	Data-Squence-Number (DSN) des Senders, die bei der Initialisieung
	pseudo-zufällig erzeugt wurde, um eins erhöht, damit der Empfänger
	doppelte Pakete erkennen und verwerfen kann.

	Es werden natürlich noch weitere Informationen wie Empfänger- und
	Sender-Adresse im Header gespeichert. Der genaue Ablauf ist für das
	Verständnis nicht weiter nötig.

\medskip

	Zum Schluss gibt der Frame -- wie im Falle des Empfangens einer
	Nachricht -- den Overhead des \ieeeframe-Frames zurück oder weist auf
	einen Fehler hin.

\minisec{Radio Driver}

	Kurz vor dem Versenden der Nachricht muss der Radio Driver diese für
	den Funksensor aufbereiten. Das Versenden geschieht \idR in zwei
	Schritten:

	Zuerst wird das Paket für die Übertragung vorbereitet und
	beispielsweise die Checksumme berechnet, aber das Paket wird noch nicht
	abgeschickt.  Danach wird das Paket erst versendet.

	Die Zweiteilung hat den Vorteil, dass zwischen Vorbereitung und
	Versendung einige Zeit vergeht.  Es ist denkbar, dass der RDC Driver
	durch ein Strobe-Paket zuerst herausfinden möchte, ob der Empfänger
	gerade erreichbr ist.  Da die Antwort ein wenig dauern kann, kann also
	in der Zwischenzeit das Paket zum Versenden vorbereitet werden, damit
	der empfangsbereite Empfänger nicht so lange warten muss, bis die
	Nachricht eintrifft.

	Da allerdings die Erläuterung des Radio Drivers sehr spezifische
	Kenntnisse des Mikrocontrollers erfordert, wird an dieser Stelle auf
	eine genaue Analyse verzichtet. Es sei nochmal wiederholt, dass der
	Funksensor meist ein Teil der Funktionen der MAC-Schicht übernehmen
	kann, die auch genutzt werden sollten, um die eigentliche
	MAC-Implementation zu entlasten
	\parensiehe{sec:NS:Funktionsweise:RadioDriver}.

\subsection{Hinweise zum Contiki NetStack}
\label{sec:ContikiNetStack:Hinweise}

	Es folgen nun noch ein paar Hinweise zum Contiki NetStack, die das
	Routing, das Funkmodul sowie den \ieeeframe-Framer betreffen:

	\begin{description}
	\item[Routing]
		Beim Emfangen von Nachrichten kann ein Routing-Knoten seine
		Nachbarschaftstabelle pflegen und hat so die Übersicht, welche
		Netzknoten in seiner Nähe noch empfangsbereit sein könnten.

		In \acs{6LoWPAN} gibt es zwei Methoden des Routings
		\autocite{Chowdhury:Route-over-VS-Mesh-under-Routing}:
		\emph{Mesh-under} und \emph{Route-over} Routing.  Die erste
		Methode routet über den \acs{6LoWPAN} Adaptation Layer und die
		zweite über den Network Layer (IPv6).

	\item[Ausschalten des Funkmoduls]
		Beim Aufruf der \lstinline=off=-Funktion kann per Parameter
		mitgeteilt werden, dass das Funkmodul doch nicht ausgeschaltet
		werden soll.  Im ersten Moment scheint diese Funktion
		überfüssig, weil man den Aufruf dieser Funktion weglassen kann.

		Da aber verschiedene Schichten miteinander -- und dies auch
		über mehr als nur eine Schicht -- interagieren müssen, ist es
		einfacher, die Einstellung einer höheren Schicht einfach
		durchzuleiten.

	\item[\ieeeframe-Framer]
		Der \ieeeframe-Frame ist in Contiki 2.6 noch nicht vollständig
		vorhanden:

		Beispielsweise fehlt der \emph{Auxiliary Security Header}
		\parencite[siehe][]{IeeeFrame:SecurityHeader:ZigBee}.  Dieser
		Header wird benötigt, wenn im Frame-Control-Feld des
		\ieeeframe-MAC-Headers das Security Enabled Bit auf 1 gesetzt
		ist.

		Das Projekt \emph{HexaBus}, welches Contiki 2.x verwendet, hat
		diesen fehlenden Header nachgerüstet
		\autocite{HexabusProject:GitHub}, aber er ist noch nicht in das
		originale Contiki Projekt aufgenommen worden.
	\end{description}
